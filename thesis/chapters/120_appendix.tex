\appendix
%\addcontentsline{toc}{section}{Anhang}
%\section*{Anhang}
\section{Anforderungen}
\subsection{Funktionale Anforderungen} \label{tab:functional-requirements}
\begin{table}[H]
    \begin{tabularx}{\textwidth}{@{}lXp{2cm}@{}}
        \toprule
        ID                & Anforderung & Quelle \\
        \midrule
        \textbf{A1.1}              & Das Gesamtsystem muss fähig sein den Lebenszyklus eines Tieres vom Erzeuger bis zum Lebensmitteleinzelhandel abzubilden.                    & \textit{Wissensch. Kontext}                \\ \addlinespace
        \multicolumn{1}{r}{A1.1.1} & Das Gesamtsystem muss fähig sein Tiere anzulegen/registrieren.                     &                 \\ \addlinespace
        \multicolumn{1}{r}{A1.1.2} & Das Gesamtsystem muss fähig sein Tiere und Chargen einander zuzuordnen.                     &                 \\ \addlinespace
        \multicolumn{1}{r}{A1.1.3} & Das Gesamtsystem muss fähig sein Tiere zwischen Teilnehmern zu transferieren im Sinne eines Eigentumswechsel.                     &                 \\
        \textbf{A1.2}              & Das Gesamtsystem muss eine generische Schnittstelle zur Kommunikation mit dem Ledger anbieten.                     & \textit{Partner}                \\ \addlinespace
        \textbf{A1.3}              & Das Gesamtsystem muss fähig sein Transaktionsdaten manipulationssicher speichern zu können.                     & \textit{Partner}                \\ \addlinespace
        \textbf{A1.4}              & Das Gesamtsystem muss fähig sein den Lebenszyklus einer Charge abzubilden.                     & \textit{Partner}                \\ \addlinespace
        \multicolumn{1}{r}{A1.4.1} & Das Gesamtsystem muss fähig sein Chargen anzulegen.                     &                 \\ \addlinespace
        \multicolumn{1}{r}{A1.4.2} & Das Gesamtsystem muss fähig sein Chargen und Tiere einander zuzuordnen.                     &                 \\ \addlinespace
        \bottomrule
    \end{tabularx}
\end{table}

\subsection{Rahmenbedingungen} \label{tab:functional-requirements}
\begin{table}[H]
    \begin{tabularx}{\textwidth}{@{}lXp{2cm}@{}}
        \toprule
        ID                & Anforderung & Quelle \\
        \midrule
        \textbf{A2.1}              & Der Prototyp muss mit der Hyperledger Fabric Blockchain-Technologie konzipiert und implementiert werden.                     & \textit{Partner}                \\ \addlinespace
        \textbf{A2.2}              & Der Prototyp bildet die Teilnehmer der Wirtschöpfungskette vom Erzeuger bis zum Lebensmitteleinzelhandel ab.                     & \textit{Partner}                \\ \addlinespace
        \textbf{A2.3}              & Der Prototyp fokussiert sich bei der Transaktionsabwicklung auf die Tierart Schwein. (Verminderte Komplexität)                     & \textit{Partner}                \\ \addlinespace
        \textbf{A2.4}\phantomsection\label{req:A2.4}              & Das Gesamtsystem muss in einer abgeschlossenen Umgebung gehosted und vor pseudonymen Zugriff geschützt sein.                     & \textit{Partner}                \\ \addlinespace
        \bottomrule
    \end{tabularx}
\end{table}

\subsection{Qualitätsanforderungen} \label{tab:functional-requirements}
\begin{table}[H]
    \begin{tabularx}{\textwidth}{@{}lXp{2cm}@{}}
        \toprule
        ID                & Anforderung & Quelle \\
        \midrule
        \textbf{A3.1}              & Die Architektur des Systems muss eine nachträgliche Erweiterung ermöglichen, um weitere Geschäftszweige abbilden zu können.                     & \textit{Wissensch. Kontext}                \\ \addlinespace
        \textbf{A3.2}              & Die Architektur des Systems muss mindestens eine konstante Performance bei steigender Teilnehmerzahl.                     & \textit{Partner}                \\ \addlinespace
        \textbf{A3.3}              & Das System muss auch bei Ausfall oder Komprimitierung eines oder mehrerer Teilnehmer konsistent und stabil weiter arbeiten.                    & \textit{Partner}                \\ \addlinespace
        \bottomrule
    \end{tabularx}
\end{table}

\newpage
\section{Listings}
\subsection{Hyperledger Fabric Peer Dockerfile} \label{lst:dockerfile-hl-peer}
\begin{lstlisting}[language=Dockerfile]
FROM golang:1.11.5

ENV DEBIAN_FRONTEND noninteractive
ENV FABRIC_ROOT=$GOPATH/src/github.com/hyperledger/fabric
ENV CHAINTOOL_RELEASE=1.1.2

# Architecture of the node
ENV ARCH=amd64
# version for the base images (baseos, baseimage, ccenv, etc.), used in core.yaml as BaseVersion
ENV BASEIMAGE_RELEASE=0.4.14
# BASE_VERSION is required in core.yaml for the runtime fabric-baseos
ENV BASE_VERSION=1.4.0
# version for the peer/orderer binaries, the community version tracks the hash value like 1.0.0-snapshot-51b7e85
# PROJECT_VERSION is required in core.yaml to build image for cc container
ENV PROJECT_VERSION=1.4.0
# generic golang cc builder environment (core.yaml): builder: $(DOCKER_NS)/fabric-ccenv:$(ARCH)-$(PROJECT_VERSION)
ENV DOCKER_NS=hyperledger
# for golang or car's baseos for cc runtime: $(BASE_DOCKER_NS)/fabric-baseos:$(ARCH)-$(BASEIMAGE_RELEASE)
ENV BASE_DOCKER_NS=hyperledger
ENV LD_FLAGS="-X github.com/hyperledger/fabric/common/metadata.Version=${BASE_VERSION} \
    -X github.com/hyperledger/fabric/common/metadata.BaseVersion=${BASEIMAGE_RELEASE} \
    -X github.com/hyperledger/fabric/common/metadata.BaseDockerLabel=org.hyperledger.fabric \
    -X github.com/hyperledger/fabric/common/metadata.DockerNamespace=hyperledger \
    -X github.com/hyperledger/fabric/common/metadata.BaseDockerNamespace=hyperledger \
    -X github.com/hyperledger/fabric/common/metadata.Experimental=true \
    -linkmode external -extldflags '-static -lpthread'"

# Peer config path
ENV FABRIC_CFG_PATH=/etc/hyperledger/fabric
RUN mkdir -p /var/hyperledger/db \
    /var/hyperledger/production \
    $GOPATH/src/github.com/hyperledger \
    $FABRIC_CFG_PATH \
    /chaincode/input \
    /chaincode/output

# Install development dependencies
RUN apt-get update \
        && apt-get install -y apt-utils python-dev \
        && apt-get install -y libsnappy-dev zlib1g-dev libbz2-dev libyaml-dev libltdl-dev libtool \
        && apt-get install -y python-pip \
        && apt-get install -y tree jq unzip\
        && rm -rf /var/cache/apt

# install chaintool
#RUN curl -L https://github.com/hyperledger/fabric-chaintool/releases/download/v0.10.3/chaintool > /usr/local/bin/chaintool \
RUN curl -fL https://nexus.hyperledger.org/content/repositories/releases/org/hyperledger/fabric/hyperledger-fabric/chaintool-${CHAINTOOL_RELEASE}/hyperledger-fabric-chaintool-${CHAINTOOL_RELEASE}.jar > /usr/local/bin/chaintool \
    && chmod a+x /usr/local/bin/chaintool

# install gotools
RUN go get github.com/golang/protobuf/protoc-gen-go \
    && go get github.com/maxbrunsfeld/counterfeiter \
    && go get github.com/axw/gocov/... \
    && go get github.com/AlekSi/gocov-xml \
    && go get golang.org/x/tools/cmd/goimports \
    && go get golang.org/x/lint/golint \
    && go get github.com/estesp/manifest-tool \
    && go get github.com/client9/misspell/cmd/misspell \
    && go get github.com/estesp/manifest-tool \
    && go get github.com/onsi/ginkgo/ginkgo

# Clone the Hyperledger Fabric code and cp sample config files
RUN cd $GOPATH/src/github.com/hyperledger \
    && git clone --single-branch -b release-1.4 --depth 1 http://gerrit.hyperledger.org/r/fabric \
    && cp $FABRIC_ROOT/devenv/limits.conf /etc/security/limits.conf \
    && cp -r $FABRIC_ROOT/sampleconfig/* $FABRIC_CFG_PATH/ \
    && cp $FABRIC_ROOT/examples/cluster/config/configtx.yaml $FABRIC_CFG_PATH/ \
    && cp $FABRIC_ROOT/examples/cluster/config/cryptogen.yaml $FABRIC_CFG_PATH/

# install configtxgen, cryptogen and configtxlator
RUN cd $FABRIC_ROOT/ \
    && go install -tags "experimental" -ldflags "${LD_FLAGS}" github.com/hyperledger/fabric/common/tools/configtxgen \
    && go install -tags "experimental" -ldflags "${LD_FLAGS}" github.com/hyperledger/fabric/common/tools/cryptogen \
    && go install -tags "experimental" -ldflags "${LD_FLAGS}" github.com/hyperledger/fabric/common/tools/configtxlator

# Install eventsclient
RUN cd $FABRIC_ROOT/examples/events/eventsclient \
    && go install \
    && go clean

# Install discover cmd
RUN CGO_CFLAGS=" " go install -tags "experimental" -ldflags "-X github.com/hyperledger/fabric/cmd/discover/metadata.Version=${BASE_VERSION}" github.com/hyperledger/fabric/cmd/discover

# The data and config dir, can map external one with -v
VOLUME /var/hyperledger
#VOLUME /etc/hyperledger/fabric

# temporarily fix the `go list` complain problem, which is required in chaincode packaging, see core/chaincode/platforms/golang/platform.go#GetDepoymentPayload
ENV GOROOT=/usr/local/go

WORKDIR $FABRIC_ROOT

# This is only a workaround for current hard-coded problem when using as fabric-baseimage.
RUN ln -s $GOPATH /opt/gopath
LABEL org.hyperledger.fabric.version=${PROJECT_VERSION} \
    org.hyperledger.fabric.base.version=${BASEIMAGE_RELEASE}
\end{lstlisting}

\subsection{Hyperledger Fabric \textit{Network Connection Profile}} \label{lst:connection-profile}
\begin{lstlisting}[language=json]
    {
        "name": "hlfv1",
        "x-type": "hlfv1",
        "x-commitTimeout": 300,
        "version": "1.0.0",
        "client": {
            "organization": "Org1",
            "connection": {
                "timeout": {
                    "peer": {
                        "endorser": "300",
                        "eventHub": "300",
                        "eventReg": "300"
                    },
                    "orderer": "300"
                }
            }
        },
        "channels": {
            "composerchannel": {
                "orderers": [
                    "orderer.example.com"
                ],
                "peers": {
                    "peer0.org1.example.com": {
                        "endorsingPeer": true,
                        "chaincodeQuery": true,
                        "ledgerQuery": true,
                        "eventSource": true
                    }
                }
            }
        },
        "organizations": {
            "Org1": {
                "mspid": "Org1MSP",
                "peers": [
                    "peer0.org1.example.com"
                ],
                "certificateAuthorities": [
                    "ca.org1.example.com"
                ]
            }
        },
        "orderers": {
            "orderer.example.com": {
                "url": "grpc://orderer.example.com:7050"
            }
        },
        "peers": {
            "peer0.org1.example.com": {
                "url": "grpc://peer0.org1.example.com:7051"
            }
        },
        "certificateAuthorities": {
            "ca.org1.example.com": {
                "url": "http://ca.org1.example.com:7054",
                "caName": "ca.org1.example.com"
            }
        }
    }
\end{lstlisting}

\subsection{Hyperledger Composer Model Definition} \label{lst:hlc-model-definition}
\begin{lstlisting}[]
namespace io.dev.foodchain


/*****************************************************************************/
/* Participant definitions                                                   */
/*****************************************************************************/

abstract participant Company identified by gln {
  o String gln
  o String name
  o Address address
  o CompanyType type
}

participant Farmer extends Company {}

participant Slaughterhouse extends Company {}

participant CuttingPlant extends Company {}

participant Manufacturer extends Company {}

participant Wholesale extends Company {}

participant Retailer extends Company {}

participant Consumer identified by consumerId {
  o String consumerId
  o String name optional
}

/*****************************************************************************/
/* Asset definitions                                                         */
/*****************************************************************************/

asset Material identified by materialId {
  o String materialId
  o MaterialType type
  o MaterialQuality quality
  o MaterialStatus status default = 'CREATED'
  o Boolean bonus optional
  --> Batch batch optional
  --> Company owner optional
  --> Company[] ownerHistory optional
  o TransportLog[] transportLog optional
  o SensorData[] sensorData optional
}

asset Batch identified by batchId {
  o String batchId
  o DateTime timestamp
  o String description optional
  o Integer availableMaterialCount default = 0
  --> Material[] materials
  o BatchStatus status
}

asset BatchNetwork identified by batchNetworkId {
  o String batchNetworkId
  o String description
  --> Batch[] nodes optional
  o Edge[] edges optional
}

/*****************************************************************************/
/* Concept & Enumeration definitions                                         */
/*****************************************************************************/

concept Address {
  o String street
  o String number
  o String postCode
  o String country
}

concept TransportLog {
  o DateTime timestamp
  o Location location
}

concept Location {
  o String latitude
  o String longitude
  o String altitude
  o String description optional
}

concept SensorData {
  o DateTime timestamp
  o String key
  o String value
}

concept Edge {
  o String edgeId
  --> Batch source
  --> Batch target
  o String weight optional
}

enum CompanyType {
  o LANDWIRT
  o SCHLACHTHOF
  o ZERLEGUNG
  o PRODUKTION
  o GROSSHANDEL
  o EINZELHANDEL
  o TRANSPORT
}

enum MaterialType {
  o SCHWEIN
  o SCHWEINEHAELFTE
  o KOPF
  o RUECKEN
  o VORDERKEULE
  o RUMPF
  o HINTERKEULE
  o SALAMI
  o BRATWURST
}

enum MaterialQuality {
  o LOW
  o MID
  o HIGH
}

enum MaterialStatus {
  o CREATED
  o REGISTERED
  o AVAILABLE_FOR_PRODUCTION
  o IN_TRANSIT
  o PROCESSED
}

enum BatchStatus {
  o CREATED
  o REGISTERED
  o AVAILABLE_FOR_PRODUCTION
  o IN_TRANSIT
  o PROCESSED
}

/*****************************************************************************/
/* Transaction definitions                                                   */
/*****************************************************************************/

transaction produceMaterial {
  o Material newMaterial // new material object
  --> Material[] inputMaterial optional // processed material (set status to 'processed')
  o Integer inputMaterialCount optional
  --> Batch[] inputBatch optional // processed batch (set status to 'processed')
  o Integer inputBatchCount optional // how many materials from this batch are used for production
}

transaction transportMaterial {
  --> Material material
  --> Company destination
}

transaction changeMaterialOwnership {
  --> Material material
  --> Company newOwner
}

transaction sellMaterial {
  --> Material material
}

transaction setMaterialStatus {
  --> Material material // material to be modified
  o MaterialStatus newStatus
}

transaction addBatchNetworkNode {
  --> BatchNetwork network // batch network to be modified
  --> Batch node
  o Edge edge optional
}

transaction addBatchNetworkEdge {
  --> BatchNetwork network // batch network to be modified
  o Edge edge
}

transaction addSensorData {
  --> Material material // material to be modified
  o SensorData data
}

transaction addTransportLog {
  --> Material material // material to be modified
  o TransportLog logEntry
}

transaction generateMockMasterData {}
transaction generateMockTransactionData {}
transaction removeMockMasterData {}
transaction removeMockTransactionData {}

/*****************************************************************************/
/* Event definitions                                                         */
/*****************************************************************************/

event changedMaterialOwnershipNotification {
    --> Material changedMaterial
}
\end{lstlisting}

\newpage
\section{Interviewguide}\label{sec:interview-guide}
\footnotesize
\textbf{1. Allgemeines}
\begin{itemize}
  \item Vorstellung vorläufige Ergebnisse Masterarbeit - Uni Oldenburg
  \item Thema \glqq Chargenrückverfolgung in der Fleischwarenindustrie\grqq
  \item Interviewverlauf
  \begin{itemize}[nosep]
    \item Kurze Einleitung: technischer / beruflicher Hintergrund Befragter
    \item Präsentation der Ergebnisse
    \item Diskussion im Anschluss
    \item Zwischenfragen / Anmerkungen jederzeit möglich
    \item Dauer: circa 30 Minuten
  \end{itemize}
  \item Informationen und Angaben werden nur für wissenschaftliche Zwecke verwendet
  \item Gespräch wird aufgezeichnet - Einverstanden?
  \item Nach dem Interview kann eine verschriftlichte Form des Interviews ausgehändigt werden
  \item Fragen?
  \item Aufnahme beginnt (\textit{Aufnahme starten})
\end{itemize}

\noindent
\textbf{2. Einleitung}
\begin{itemize}
  \item Beschreibe deine aktuelle Tätigkeit
  \item Inwiefern spielen innovative Ideen in deinem Beruf eine Rolle?
\end{itemize}

\noindent
\textbf{3. Präsentation Ergebnisse}\\
\noindent
\textbf{4. Diskussion}
\begin{itemize}
  \item Was ist dein erster Gedanke zum gerade präsentierten Prototyp?
  \item Wie denkst du über das Konzept, die Chargenrückverfolgung über eine Blockchain abzuwickeln?
  \item Wie ist deine Einschätzung zur Implementierung?
  \item Was hälst du vom gewählten Anwendungsfall bzw. den Rahmenbedingungen?
  \item Wie schätzt du das Potential des gezeigten Prototypen ein?
  \item Fallen dir weitere Anwendungsfälle ein?
\end{itemize}

\noindent
\textbf{5. Ende des Interviews}
\begin{itemize}
  \item Information: Interview ist vorbei, Aufnahme wird gestoppt (\textit{Aufnahme stoppen})
  \item Vielen Dank für die Teilnahme
  \item Sollten noch Fragen aufkommen, bitte kontaktieren
\end{itemize}
\normalsize

\newpage
\section{Transkription Experteninterview}\label{sec:interview}
\begin{table}[ht!]
  \begin{tabular}{ll}
  Datum und Ort:           & 14.10.2019                                                                                                                                                                                                                       \\
  Dauer:                   & 31:24 Minuten                                                                                                                                                                                                                    \\
  Interviewer:             & Nils Lutz                                                                                                                                                                                                                        \\
  Datum der Transkription: & 15.10.2019                                                                                                                                                                                                                       \\
  Transkriptionsregeln:    & \begin{tabular}[c]{@{}l@{}}Vollständige Sprachglättung\\ Pausenlänge in Sekunden: (.)/(..)/(...)\\ Nonverbale Äußerungen: (\textit{lachen})\\ Nicht-sprachliche Ereignisse: ((\textit{Unterbrechung}))\\ \textbf{I:} Interviewer, \textbf{B:} Befragter\end{tabular}
  \end{tabular}
\end{table}

\begin{itemize}
  \setlength\itemsep{0em}
  \linenumbers
  \item[\textbf{I:}] Guten Morgen, kannst du vielleicht einfach mit einer Beschreibung deiner Position im Konzern beginnen?
  \item[\textbf{B:}] Ja klar, also ich habe bei Westfleisch angefangen als IT-Koordinator in einem unserer Produktionswerke in Coesfeld und bin aktuell bei Westfleisch stellvertetender IT-Leiter auf Konzernebene. Zum Verständnis Westfleisch ist ein Zusammenschluss aus vielen einzelnen Unternehmen mit jeweils eigenen IT-Abtei\-lungen. Es gibt Produktionswerke für die verschiedenen Produkte die wir anbieten, Transport- und Logistik Unternehmen um die Roherzeugnisse innerhalb der Unternehmensgruppe zu bewegen, sowie Finanzverwaltungsunternehmen. Meine Aufgabe ist es auf strategischer Ebene die IT-Abteilungen all dieser einzelnen Unternehmen zentral zu steuern, damit wir die, von der Konzernleitung definierten, Ziele erreichen können. Außerdem bin ich der Projektleiter auf unserer Seite für das Forschungsprojekt REIF, was ja in Zusammenarbeit mit der Firma CompanyMind, Jade Hochschule, TU München und dem Frauenhofer Institut offiziell im September vom BMWi ausgezichnet worden ist.
  \item[\textbf{I:}] Danke, ich würde dir dann jetzt erstmal gerne meine Ergebnisse zeigen und anschließend mit dir in eine offene Diskussion gehen, um dein Feedback dazu mitnehmen zu können.
  \item[\textbf{B:}] Klingt gut, dann zeig mal her!
  \item[\textbf{I:}] Moment. ((\textit{Öffnen der Präsentation})) Also Arbeitstitel ist \glqq Chargenrückverfolgung in der Fleischwarenindustrie - Konzeption und prototypische Implementierung einer Blockchain Lösung\grqq{}. Ich habe mich also damit beschäftigt ob es möglich ist die Chargenrückverfolgung über eine Blockchain abzubilden. Ich bin zu dem Thema grundsätzlich über die Technologie gekommen. Das heißt ich wollte was mit der Blockchain machen und habe mir dazu Probleme aus der Wirtschaft gesucht die man eventuell mit einer Blockchain besser lösen könnte als mit bisherigen Lösungen. Und da wir ja schon etwas länger mit euch zusammenarbeiten innerhalb der App-Entwicklung war es für mich halt naheliegend einen Use-Case aus der Fleischwarenindustrie zu nehmen. Mir war ja bekannt, das ihr bereits SAP Global Track \& Trace im Einsatz habt zum Chargenmanagement und SAP ebenfalls auf den Blockchain Zug aufgesprungen ist. Das war so die Motivation im Grunde wieso die Konstellation Blockchain, Fleischwarenindustrie, Westfleisch zustande gekommen ist. Man hört ja immer wieder mal in den Medien, dass Produkte verunreinigt sind und es große Rückrufaktionen gibt. Letztendlich habe ich dann bei meiner Recherche herausgefunden, dass solche Rückrufaktionen teils ziemlich lange dauern in der Vorbereitung. Also es gibt Fälle wo zum Beispiel eine verunreinigte Snickers Charge knapp 4 Wochen noch im Umlauf war, einfach weil es so lange gedauert hat rauszufinden a) wo die Verunreinigung begonnen hat und b) in welche Produktionschargen diese Verunreinigung dann weiter getragen wurde.
  \item[\textbf{B:}] Okay, so weit konnte ich folgen ((\textit{lachen})). Thema Blockchain (..) ist ja in aller Munde zur Zeit. Damit haben wir auch Berührungspunkte im Forschungsprojekt, da wir dort auch versuchen die Wertschöpfungskette der Lebensmittelbranche zu optimieren. Bevor wir da jetzt tiefer einsteigen, kannst du mir einen kurzen Abriss zur Technologie geben?
  \item[\textbf{I:}] Ja klar, habe ich sowieso in der Präsentation mit drin auf Grund der Aktualität des Themas. ((\textit{Folienwechsel})) Ich versuch das mal so kurz wir möglich zu halten, wenn irgendwas noch unklar sein sollte einfach eben zwischenfragen. Also du kannst dir eine Blockchain grundsätzlich erstmal als Datenbank vorstellen. Diese Datenbank garantiert dir jetzt, dass alle Datensätze die darin erfasst wurden zu keinem Zeitpunkt mehr ungewollt verändert werden können. Zusätzlich verspricht die Technologie eine Art Failsafe Betrieb. Das heißt es ist wie ein Cluster DBMS zu verstehen, fällt ein Knoten des Systems aus ist dadurch nicht das Gesamtsystem betroffen und es arbeitet weiter. Dann hast du sicherlich schon von Smart Contracts gehört. Diese musst du dir vorstellen wie eine \glqq Stored Procedure\grqq{} in einer Datenbank, nur das du ein paar mehr Möglichkeiten hast Geschäftslogik darin auszudrücken. So weit ist das nichts neues, der Clou ist jetzt aber das so ein System im besten Fall vollständig dezentral aufgestellt ist. Das heißt diese \glqq Datenbank\grqq{} wird nicht zentral bei Westfleisch betrieben, sondern es wird ein Netzwerk mit den verschiedenen Teilnehmer der Wertschöpfungskette aufgespannt. Ich hab mich ja für die Chargenrückverfolgung entschieden, entsprechend besteht mein konzipiertes Netzwerk aus Landwirten, Mästern, Produktionswerken und dem Groß- bzw. Einzelhandel. Jeder dieser Teilnehmer betreibt mindestens einen Knoten auf dem das Blockchain System arbeitet. Wenn jetzt ein neuer Datensatz erfasst werden soll, sagen wir mal ein Landwirt will seine Schweine zur Schlachtung anmelden. Dann spricht man bei der Blockchain von Transaktionen. Mit dieser Transaktion wird dem Blockchain System gesagt, es soll ein neuer Datensatz mit den erfassten Informationen angelegt werden. Der Systemknoten des Landwirts verschickt diese Transaktion dann an alle Knoten im Netzwerk und lösen dadurch den Smart Contract, also die Geschäftslogik aus. Alle Knoten prüfen dann ob die Transaktion nach der Geschäftslogik valide ist und stimmen dann darüber ab, ob der Datensatz hinzugefügt werden soll. Gehen wir mal davon aus, die Transaktion war valide und wurde hinzugefügt. Jetzt ist es nicht mehr möglich das ein einzelner Knoten nachträglich Werte des Datensatz verändert ohne das alle anderen Knoten davon etwas mitbekommen würden, dazu verwendet die Blockchain Hashwerte und digitale Signaturen wie du es aus der kryptographischen Verschlüsselung kennst. Sprich, sollte ein Knoten doch etwas an einem Datensatz verändern könnten alle anderen Teilnehmer dies sofort herausfinden in dem sie einfach die Hashwerte der Datensätze miteinander vergleichen und feststellen würden, dass da etwas nicht passt. Daher kommt eigentlich auch der Begriff Blockchain. Alle Datensätze werden in einer Kette aus Blöcken gespeichert und der aktuellste Block referenziert immer auf den Hashwert des vorigen Blocks.
  \item[\textbf{B:}] Macht Sinn, Hashwerte sagen mir noch was aus meinem Studium ((\textit{lachen})) auch wenn das schon etwas länger her ist. (...) Wie hast du das jetzt angepasst, um es für eine Chargenrückverfolgung nutzbar zu machen?
  \item[\textbf{I:}] ((\textit{Folienwechsel})) Also angefangen habe ich mit der Suche nach einem geeigneten Blockchain System für den industriellen Einsatz. Man kennt ja Bitcoin, aber die Blockchain von Bitcoin eignet sich in diesem Fall eher weniger, da es eine öffentliche Blockchain ist in der jeder als Netzwerkteilnehmer mitmachen kann und alle Transaktionen öffentlich einsehbar sind. Zusätzlich ist die Verarbeitungsgeschwindigkeit bei vollständig öffentlichen Blockchains aktuell noch vergleichsweise langsam. Da sind konventionelle Datenbanksysteme um ein vielfaches schneller. Es gibt allerdings auch Blockchain Systeme die speziell für den Einsatz in der Industrie entwickelt wurden. Dazu gehört unter anderem Hyperledger Fabric. Das wurde ursprünglich mal von IBM konzipiert und entwickelt und dann als Open-Source Software an die Linux Foundation übergeben. Dadurch kann zum Beispiel jeder in den Quelltext gucken und sicherstellen, das das System wirklich so arbeitet wie es angepriesen wurde. Außerdem sind Blockchain Netzwerke die mit Hyperledger Fabric gebaut wurden in der Regel nicht öffentlich und es gibt klar definierte Regeln wer als Teilnehmer in dem Netzwerk auftreten kann. Es ist nicht zwingend notwendig, das jeder Teilnehmer auch immer die Transaktionen validiert. Ein Unternehmen kann zum Beispiel drei Knoten im Netzwerk betreiben, aber nur zwei Knoten validieren neue Transaktionen und der dritte Knoten sorgt einfach für mehr Ausfallsicherheit für das gesamte Netzwerk.
  \item[\textbf{B:}] Also hast du dich für dieses Hyperledger entschieden auf Grund der Geschwindigkeit und dem Fokus der Software auf den industriellen Sektor?
  \item[\textbf{I:}] Ja genau, die Geschwindigkeit war ein ausschlaggebender Punkt. Dazu war es für mein Konzept ebenfalls wichtig das ich die Geschäftslogik möglichst in einer Sprache implementieren kann die ich auch verstehe ((\textit{lachen})). Und natürlich der Aspekt das Hyperledger Fabric ein private Blockchain System ist, wo ich kontrollieren kann wer alles im Netzwerk mitmachen kann. Zusätzlich gibt es für Hyperledger Fabric ein Framework das sich Hyperledger Composer nennt, mit dem man dann die Smart Contracts entwickeln kann die dann später in das Netzwerk installiert werden. Ich würde dir als nächstes einfach mal ein kurzes Beispiel zeigen wie so eine Transaktion im Netzwerk abläuft und welche Zugriffsmöglichkeiten man dann hat, um mit dem System als Anwender bzw. aus anderen Anwendungen heraus zu interagieren. ((\textit{Demo am System}))
  \item[\textbf{B:}] Okay, das ist ziemlich interessant. Mir kommen da auch direkt ein paar Einsatzgebiete bei uns ((\textit{lachen})) aber erzähl du erstmal weiter.
  \item[\textbf{I:}] Ja im Grunde würde ich jetzt gerne in die offene Diskussion über gehen, um einfach mal von dir zu hören was du so darüber denkst und wie du grad schon meintest wo man es vielleicht noch einsetzen könnte.
  \item[\textbf{B:}] Also erstmal muss ich sagen, das sieht für einen Prototypen schon relativ ausgereift aus, wenn man es jetzt auf den Use-Case Chargenrückverfolgung für Schweine belässt. Generell denke ich, ich habe den Ansatz dahinter verstanden. Würden wir dieses System jetzt bei Westfleisch einsetzen, müssten wir nur gucken, das wir alle unsere Zulieferer und Kunden mit ins Boot holen, das die sich alle so einen Blockchain Knoten hinstellen, richtig?
  \item[\textbf{I:}] Genau, es lässt sich in der Theorie auch mit nur einem Knoten betreiben, dann geht aber der Sinn eines dezentralen Systems verloren. Daher habe ich es auch so konzipiert, dass jeder Teilnehmer im Minimum einen Knoten betreiben muss, um die Eigenschaften der Blockchain ausreizen zu können.
  \item[\textbf{B:}] Absolut! (..) Also können wir über die Smart Contracts unsere Geschäftslogik abbilden bzw. auch unsere Zulieferer und Endkunden. Wenn man bedenkt das wir so knapp 130 Kunden haben und jeder Kunde uns ein anderes Format für ihre Chargeninformationen vorgibt bzw. nutzt, dann würde so ein System schon wirklich Sinn machen allein aus Gründen der Standardisierung. Ich habe ja auch das Forschungsprojekt REIF im Hinterkopf bei der ganzen Geschichte. Da versuchen wir ja auch mit KI und maschinellem Lernen unsere Produktionsplanung zu optimieren. Wenn jetzt die Landwirte in der Lage sind über dein System ihr Vieh bei uns anzumelden und wir mit dem maschinellem Lernen dann ebenfalls auf die Daten zugreifen könnten (...) ja dann erhöht sich zum einen bei uns die Planungssicherheit und zum anderen könnten wir Informationen zu den Chargen bzw. Produkten in beide Richtungen der Wertschöpfungskette bereitstellen mit der Garantie das diese Daten der Wahrheit entsprechend. Das bringt ja Vertrauen und grade der Endverbraucher wird immer kritischer und möchte soviele Informationen zu seinem Produkt wie möglich haben und sich dabei auch sicher sein, das die Unternehmen ihm da die Wahrheit erzählen.
  \item[\textbf{I:}] Da unterstützt dich die Technologie dann mit den kryptographischen Methoden wie die digitalen Signaturen. Sowas kennen viele ja bereits von Websiten wenn sie das kleine grüne Schloss neben der URL sehen. Dann wissen sie das diese Seite sicher ist, weil sie sich indirekt darauf verlassen können das die Zertifikate offiziell beglaubigt wurden sozusagen. Ich mein für die meisten reicht sicher das grüne Schloss und was da im Hintergrund läuft ist nicht so wichtig, aber wenn dann doch mal etwas ist grade bei Lebensmitteln oder zum Beispiel Finanzen möchte man als Endanwender ja doch eine gewisse Sicherheit haben das alles mit rechten Dingen zugeht. Kannst du vielleicht noch etwas zur Implementierung und den damals gewählten Rahmenbedingungen sagen?
  \item[\textbf{B:}] Also ich hab gesehen das du die Oberflächen mit Fiori modelliert hast. Das kenne ich ja von den anderen Apps, die ihr bereits für uns entwickelt habt. Da könnte man sich sicherlich nochmal mit den Fachabteilungen hinsetzen und gucken das man da einen Feinschliff reinbekommt. Ich mein, für einen Prototyp ist das aus meiner Sicht völlig ausreichend, aber wenn man sowas dann auf einer Messe präsentieren möchte vielleicht im Zusammenspiel mit einem KI Systen ((\textit{lachen})) dann muss sowas ja heutzutage alles sehr gut aussehen. Sonst kriegt man die Kunden nicht abgeholt. Zu deiner Implementierung im Backend kann ich nicht ganz soviel sagen, die Datenmodelle sehe schlüssig aus und die Demo hat ja gezeigt, dass man damit den Geschäftsprozess komplett abbilden kann. Da würde ich sagen: Passt! Die Rahmenbedingungen hattest du glaub ich zu Beginn deiner Arbeit mit unserer Infrastruktur Abteilung abgestimmt richtig?
  \item[\textbf{I:}] Genau, so war das. Ich wollte halt die Rahmenbedingungen so definieren, das das System nicht nach der Konzeption bzw. der prototypischen Implementierung in der Schublade verschwindet. Deswegen war mir wichtig da schon früh eine Art Sparringspartner aus der Industrie zu haben.
  \item[\textbf{B:}] Ja dann denke ich das die Rahmenbedingungen schon erfüllt sind, sicher lassen sich noch weitere finden wenn man das System vom Prototypenstatus in einen Pilotbetrieb überführen will, aber das war ja nicht Bestandteil deiner Arbeit. Insofern passt das.
  \item[\textbf{I:}] Also würdest du sagen der Prototyp besitzt ein gewisses Potential?
  \item[\textbf{B:}] Definitiv! (..) Der entscheidende Punkt ist aus meiner Sicht ist oft die Marktdurchdringung. Du kannst noch so tolle Systeme und Technologien entwickeln, wenn niemand am Markt oder in der Branche dieses System nutzt, aus welchen Gründen auch immer, dann wird dieses System keinen Erfolg haben. Deshalb sollte man im Blick behalten, das mit so einem System eine Art \glqq Win-Win\grqq{} Situation hergestellt wird. Wenn ich als Teilnehmer des Netzwerk etwas hineingebe muss ich auch immer etwas herausbekommen, sonst sinkt mein Interesse dieses System zu verwenden. Und wie du erklärt hast, steigt die Sicherheit und der Informationsgehalt etc. mit der Anzahl der Teilnehmer. Ich denke da immer an das Beispiel mit den Elektroautos. Die Technologie ist super, aber wenn wir kein Ladenetz haben wird niemand sich ein Elektroauto kaufen. So einfach ist. ((\textit{lachen}))
  \item[\textbf{I:}] Danke (..) du hattest am Anfang schon gesagt das dir direkt weitere Anwendungsfälle in den Kopf kommen. Kannst du darüber noch was sagen? Danach wäre ich auch durch mit meinen Fragen. ((\textit{lachen}))
  \item[\textbf{B:}] Natürlich, wie gesagt generell könnte man so ein System für sämtliche Tierarten erweitern, die wir so durch die Produktionswerke schieben. Obendrauf wäre es ziemlich interessant Auswertungen der Veterinäre mit zu erfassen. Endkunden wollen wissen wieviel Antibiotika in ihrer Wurst steckt. Grade bei Hühnerfleisch, da legen die Käufer sehr viel Wert drauf mittlerweile. Außerdem wird immer öfter nicht nur auf die Art und Weise der Haltung geschaut, sondern auch was die Tiere während ihres Lebens als Futter bekommen haben. Das geht soweit, das solche Angaben einen höheren Preis des Endprodukts vollständig rechtfertigen und der Markt für solche Produkte ist da. Das lässt sich nicht mehr bestreiten. Aus Sicht der Wertschöpfungskette, also sagen wir mal aus Sicht eines Großhandels werden Transport- und Logistikinformationen immer wichtiger. Ich will wissen, wann und wie lange waren die Tiere unterwegs und wie ist ihr Zustand während des Transports. Sowas wird aktuell schon über ein paar wenige Sensoren und die Prüfung durch den Spediteur ermittelt. Könnte man diese Informationen jetzt noch mit in die Blockchain packen hätte man alles an einem Ort, was widerrum Futter für die KI bzw. die Algorithmen des maschinellen Lernen ist ((\textit{lachen})) Ich denke da lässt sich noch einiges mit machen, wenn man so ein System im Sinne der Industrie 4.0 einsetzt und möglichst viele Informationen dort reinspeichert und sich stets sicher sein kann das dort nachträglich niemand an den Daten rumpfuschen kann. Demnach hast du mit deiner Arbeit ein ganzen Stück an Vorarbeit für das Forschungsprojekt REIF geleistet und bewiesen das eine Rückverfolgbarkeit mit dieser Technologie vom Landwirt bis zum Endkunden machbar ist.
\end{itemize}

\newpage
