\section{Grundlagen} \label{sec:basics}
In diesem Kapitel sollen zuerst die Grundlagen zur \textit{Chargenrückverfolgung} selbst dargelegt werden. Dies erfolgt über eine allgemeine Definition einer \textit{Charge} einer anschließenden Einordnung von \textit{Chargen} in die Wertschöpfungskette der Fleischwarenindustrie sowie den besonderen Dokumentationspflichten für \textit{Chargen} in Deutschland. Darüber hinaus wird in Kapitel \ref{sec:blockchain-technology} die \textit{Blockchain-Technologie} erörtert. Hier wird ebenfalls eine grundsätzliche Definition der Technologie gegeben, sowie eine Abgrenzung zwischen den verschiedenen Begrifflichkeiten vorgenommen. Außerdem wird im Detail auf die einzelnen technischen Konzepte und Komponenten eingegangen, aus denen sich eine \textit{Blockchain} bildet. So soll ein grundlegendes Verständnis für die beiden Thematiken \textit{Chargenrückverfolgung} und \textit{Blockchain-Technologie} aufgebaut werden als Unterstützung für das darauf folgende Lösungskonzept und den Systementwurf.

% Chargenrückverfolgung
\subsection{Chargenrückverfolgung} \label{sec:batch-traceability}
\textcolor{red}{Notwendigkeit einer Charge erläutern auf Grund der Gruppierung von vielen Einzelprodukten eben zu einer Charge.}

\subsubsection{Definition Charge}

Eine \textit{Charge} bezeichnet eine Ansammlung eines Produkts, welche unter gleichen Bedingungen produziert wurde. Bei dem Produkt kann es sich beispielsweise um Werkstoffe, Bauteile, Baugruppen oder Endprodukte handeln. Die Begriffe \textit{Los} oder \textit{Partie} werden oft als Synonym für \textit{Charge} verwendet. Einige Branchen sind bei der Produktion auf die Erzeugung definierter \textit{Chargen} zugeschnitten. Diese Chargenproduktion, die auch diskontinuierliche Produktion genannt wird, zeichnet sich durch einen zeitlich unterbrochenen Materialfluss aus. So kann ein Produktionsgefäß mit unterschiedlichen Rohstoffen befüllt und anschließend verarbeitet werden. In der diskontinuierlichen Produktion versteht man daher unter einer \textit{Charge} eine Menge eines Erzeugnisses, welche in einem Produktionsgang gefertigt worden ist und identische Kennzeichen in Bezug auf Materialzusammensetzung, Fertigungsprozess und Produktqualität aufweist. Beispiele hierfür finden sich in der Stahlproduktion, der pharmazeutischen und chemischen sowie in der Lebensmittelindustrie \citep{Guenther2012}.

Inzwischen wird der Begriff der \textit{Charge} aber auch in der kontinuierlichen Produktion verwendet. Die \textit{Charge} wird dabei durch die Berücksichtigung einer oder mehrerer der folgenden Eigenschaften charakterisiert:

\begin{itemize}
  \item Herstellung auf einer Fertigungslinie,
  \item einheitliche Zulieferteile,
  \item homogene Qualität,
  \item gleichbleibende Prozesskette,
  \item identisches Produktionsdatum.
\end{itemize}

Es bleibt festzuhalten, dass die Parameter in der kontinuierlichen Produktion nicht so eindeutig abgrenzbar sind wie in der diskontinuierlichen Produktion. Zudem können in der kontinuierlichen Produktion Schwankungen durch dynamische Prozesse wie Abnutzung von Werkzeugen auftreten, die innerhalb einer definierten \textit{Charge} zu deutlichen Qualitätsunterschieden führen können und so die Praxistauglichkeit der Chargenverfolgung in Frage stellen.

In der für die Lebensmittelindustrie wichtigen \ac{lkv} wird unter einem \textit{Los} \glqq die Gesamtheit von Verkaufseinheiten eines Lebensmittels verstanden, das unter praktisch gleichen Bedingungen erzeugt, hergestellt oder verpackt wurde.\grqq{} \citep{LKV1993}. Dagegen bezeichnen laut Code of Federal Regulation \textit{Los} oder \textit{Charge} \glqq ein oder mehrere Bauteile oder fertige Geräte eines einzigen Typs, Version, Klasse, Größe, Zusammensetzung oder Software Version, welche im wesentlichen unter gleichen Bedingungen hergestellt werden und die innerhalb spezifizierter Grenzen einheitliche Eigenschaften und Qualität haben sollen.\grqq{} \citep{QSR1996}. Somit können auch einzelne Produkte eine \textit{Charge} oder ein \textit{Los} bilden. Im Hinblick auf eine möglichst genaue Eingrenzung bestimmter Produkte beispielsweise bei einer Rückrufaktion sollte eine kleinstmögliche Chargengröße gewählt werden, die im Idealfall nur ein einzelnes Produkt umfasst.

\subsubsection{Einordnung in die Wertschöpfungskette}

Die Chargenverfolgung wird innerhalb des Produktionsprozesses für das Upstream Tracing und in dem Distributionsprozess für das Downstream Tracing eingesetzt. Bei einer gut organisierten Chargenverfolgung im Downstream Prozess behält der Hersteller den Überblick, wo seine Produkte wann gelagert, verkauft und eingesetzt werden und ist so in der Lage, gezielt Rückrufe durchzuführen. Durch die Chargenverfolgung im Upstream Prozess können eventuelle Qualitätsprobleme bis zum Vorlieferanten nachverfolgt werden. Abbildung \ref{fig:wkd-Lebensmittelindustrie} zeigt schematisch die Wertschöpfungskette in der Lebensmittelindustrie. Bei einem optimal eingerichtetem Up- und Downstream Tracing behalten die Hersteller und Konsumenten während der ganzen Wert"-schöpfung einen Überblick wo sich die Waren aktuell im Einsatz befinden.

\begin{figure}[h!]
	\centering
	\includegraphics[width=1.0\linewidth]{pictures/system-of-agribusiness}
	\caption[Wertschöpfungskette: Lebensmittelindustrie]{Wertschöpfungskette: Lebensmittelindustrie \citep{Strecker2010}}
	\label{fig:wkd-Lebensmittelindustrie}
\end{figure}

\paragraph{Downstream Tracing (Abwärts-Rückverfolgbarkeit)}$~~$\\
Als Downstream Tracing wird die Rückverfolgbarkeit ausgehend vom Erzeuger zum Endprodukt bezeichnet. Gegenstand der Rückverfolgung ist typischerweise ein \textit{Los} (\textit{Charge}) oder eine einzelne Einheit eines Produkts. Abhängig vom Grad der Integration innerhalb der Lieferkette lässt sich die Rückverfolgung bis zum Einzelhandel bzw. auch bis zum Endverbraucher durchführen. Zum Einsatz kommt das Downstream Tracing wenn Probleme in Waren zu einem späten Zeitpunkt festgestellt wurden und geprüft werden muss in welchen Endproduktchargen sich hierdurch weitere Probleme ergeben könnten \citep{Trienekens2001, Zailani2010}. \citet{Wegner-Hambloch2004} beschreibt Downstream Tracing als \glqq Ortsbestimmung von bereits hergestellten Produkten zwecks nachträglichen Rückrufs von gesundheitsgefährdenden Produkten\grqq{}.

\paragraph{Upstream Tracing (Aufwärts-Rückverfolgbarkeit)}$~~$\\
Unter Upstream Tracing versteht man die Rückverfolgbarkeit vom Endverbraucher in Richtung des Erzeugers. Tritt ein Problem bei Lebensmittelprodukten auf wird das Upstream Tracing zur Ursachenforschung eingesetzt. So lassen sich Probleme die beispielsweise vom Konsumenten beim Endprodukt oder bei einer Qualitätskontrolle von Teilprodukten festgestellt wurden zurückverfolgen bis zum Urerzeuger \citep{Trienekens2001, Zailani2010}. Nach \citet{Wegner-Hambloch2004} ist Upstream Tracing \glqq die Bestimmung der Produktgeschichte vom Endprodukt [...] bis zu den Futtermitteln.\grqq{}

%\subsubsection{Zentrale vs. dezentrale Ansätze}
%\textcolor{red}{Unterschied zwischen zentraler Informationssysteme (F-Trace) und dezentraler logischer Systeme (Zugriff auf F-Trace). Letzteres sind nur dem Anschein nach dezentral. Ihre zugrunde liegende Infrastruktur der Informationssysteme ist zentral und wird von einem Intermediär verwaltet und betrieben. Angriffspunkte für Manipulation und Kontrolle eines einzelnen rausarbeiten. \citep{Steins2015, allgemeinefleischerzeitung2011}
%Lorem ipsum dolor sit amet, consetetur sadipscing elitr, sed diam nonumy eirmod tempor invidunt ut labore et dolore magna aliquyam erat, sed diam voluptua. At vero eos et accusam et justo duo dolores et ea rebum. Stet clita kasd gubergren, no sea takimata sanctus est Lorem ipsum dolor sit amet. Lorem ipsum dolor sit amet, consetetur sadipscing elitr, sed diam nonumy eirmod tempor invidunt ut labore et dolore magna aliquyam erat, sed diam voluptua. At vero eos et accusam et justo duo dolores et ea rebum. Stet clita kasd gubergren, no sea takimata sanctus est Lorem ipsum dolor sit amet.Lorem ipsum dolor sit amet, consetetur sadipscing elitr, sed diam nonumy eirmod tempor invidunt ut labore et dolore magna aliquyam erat, sed diam voluptua. At vero eos et accusam et justo duo dolores et ea rebum. Stet clita kasd gubergren, no sea takimata sanctus est Lorem ipsum dolor sit amet. Lorem ipsum dolor sit amet, consetetur sadipscing elitr, sed diam nonumy eirmod tempor invidunt ut labore et dolore magna aliquyam erat, sed diam voluptua.}

\subsubsection{Dokumentationspflichten}
Für landwirtschaftliche Waren und daraus hergestellte Nahrungsmittel existieren eine Vielzahl von gesetzlichen Regelungen aus denen Bedingungen und Anforderungen zum Thema Rückverfolgbarkeit abgeleitet werden können. Die VO (EG) Nr. 178/02 \citep{EPER2002} wird in diesem Kontext als Basisverordnung gesehen. Darüber hinaus sind die horizontale Lebensmittelhygieneverordnung sowie die vertikalen Hygieneverordnungen für Fleisch und Fleischerzeugnise, Milch- und Milcherzeugnisse, Fisch und Fischerzeugnisse mit der Vorgabe zur Umsetzung betrieblicher Eigenkontrollen oder Einrichtung eines \acs{haccp}-Systems\footnote{Englisch für \textit{\acf{haccp}}. Beschreibt ein Qualitätskontrollsystem für den sicheren Umgang mit Lebensmitteln durch strukturierte und präventive Maßnahmen zur Verhinderung von Erkrankungen und Verletzungen des Konsumenten.\citep{EPER2004}} elementare Bestandteile eines wirkungsvollen, innerbetrieblichen Rückverfolgungssystems in Lebensmittelbetrieben. Eine verbindliche fünfjährige Speicherung von Daten der Transaktionen bezüglich der Lieferanten und Abnehmer ist ebenfalls festgelegt.\\

\noindent
Weitere Regelungen zur Rückverfolgbarkeit für die EU:
\begin{itemize}
  \item Rindfleischetikettierungs-VO (EWG) Nr. 1760/2000
  \item EU-Öko-VO (EWG) 2092/91
  \item EU-Verordnung über amtliche Futter- und Lebensmittelkontrollen (Vorschlag vom 5. Februar 2003)
  \item Vermarktungsnormen für Eier 1907/90/EWG
\end{itemize}
Nationale Regelungen für Deutschland:
\begin{itemize}
  \item \acf{lmkv}
  \item \acf{lkv}
  \item verschiedene Fleisch- und Geflügelfleisch-Hygienevorschriften
  \item Weingesetz und Weinwirtschaftsgesetz
  \item Handelsklassenrecht
  \item \acf{lmbg}
\end{itemize}
Über die gesetzlichen Regelungen hinaus gelten verbindliche Standards der Handelsseite, die übergreifend von der \ac{gfsi} vorgegeben werden. Der in Deutschland meist gefragte \ac{ifs}, der Standard des \ac{brc} für Lieferanten nach England und diverse andere Standards definieren das detaillierte Anforderungsniveau transparenter Warenströme aus Handelssicht für den Hersteller.

%\subsubsection{???Besonderheiten der Fleischwarenindustrie???}
%\textcolor{red}{Lorem ipsum dolor sit amet, consetetur sadipscing elitr, sed diam nonumy eirmod tempor invidunt ut labore et dolore magna aliquyam erat, sed diam voluptua. At vero eos et accusam et justo duo dolores et ea rebum. Stet clita kasd gubergren, no sea takimata sanctus est Lorem ipsum dolor sit amet. Lorem ipsum dolor sit amet, consetetur sadipscing elitr, sed diam nonumy eirmod tempor invidunt ut labore et dolore magna aliquyam erat, sed diam voluptua. At vero eos et accusam et justo duo dolores et ea rebum. Stet clita kasd gubergren, no sea takimata sanctus est Lorem ipsum dolor sit amet.Lorem ipsum dolor sit amet, consetetur sadipscing elitr, sed diam nonumy eirmod tempor invidunt ut labore et dolore magna aliquyam erat, sed diam voluptua. At vero eos et accusam et justo duo dolores et ea rebum. Stet clita kasd gubergren, no sea takimata sanctus est Lorem ipsum dolor sit amet. Lorem ipsum dolor sit amet, consetetur sadipscing elitr, sed diam nonumy eirmod tempor invidunt ut labore et dolore magna aliquyam erat, sed diam voluptua. At vero eos et accusam et justo duo dolores et ea rebum. Stet clita kasd gubergren, no sea takimata sanctus est Lorem ipsum dolor sit amet. Lorem ipsum dolor sit amet, consetetur sadipscing elitr, sed diam nonumy eirmod tempor invidunt ut labore et dolore magna aliquyam erat, sed diam voluptua. At vero eos et accusam et justo duo dolores et ea rebum. Stet clita kasd gubergren, no sea takimata sanctus est Lorem ipsum dolor sit amet.}


% \textit{Blockchain-Technologie}
\subsection{\textit{Blockchain-Technologie}}

\subsubsection{Definition} \label{blockchain-definition}
Eine \textit{Blockchain} als Ganzes betrachtet, ist ein System zur Transaktionsabwicklung mit besonderen Eigenschaften. Als erstes beschrieben wurde die \textit{Blockchain} im Paper von \cite{Nakamoto2009} zur Realisierung der digitalen Währung \ac{btc}. Aus technischer Sicht gehört die \textit{Blockchain-Technologie} zum Bereich der verteilten Datenbanken. Ein \textit{Block} in einer \textit{Blockchain} repräsentiert eine Menge von Datensätzen die in der \textit{Blockchain} (Datenbank) vorgehalten werden. Jeder \textit{Block} (Datensatz) widerrum besitzt genau einen Vorgänger und einen Nachfolger. Allerdings werden diese Blöcke nicht wie in klassischen relationalen Datenbanksystemen in Tabellenstrukturen abgelegt und verwaltet. Durch die Vorgänger Information wird jeder neue Datensatz immer an den letzten Datensatz angehangen. Daraus bildet sich eine Kette von Blöcken - daher der Name \textit{Blockchain} (dt. Blockkette).

Ein \textit{Block} innerhalb der Kette kann definiert werden als verschlüsseltes Stück Information. Er beinhaltet neben den Transaktionen noch einen Zeitstempel und zwei kryptographische Hashwerte. Der erste Hashwert wird aus dem \textit{Block} selbst gebildet und der zweite Hashwert ist die Verknüpfung zum Vorgänger \citep{Tschorsch2016}. Wird nachträglich ein Wert einer Transaktion verändert oder ein ganzer \textit{Block} aus der Kette entfernt passt der jeweilige Hashwert des Vorgängers nicht mehr und durch den linearen Aufbau der \textit{Blockchain} würde diese Manipulation jederzeit unmittelbar bemerkt werden bei der Validierung von neuen Transaktionen. Die Daten in der \textit{Blockchain} sind somit vor unbefugter Veränderung geschützt. Als dezentrale Datenbank wird auf jedem Knoten des sich aufspannenden Netzwerks aus Teilnehmern der \textit{Blockchain} eine exakte Kopie\footnote{Es gibt Ausprägungen von \ac{dlt} Systemen bei denen sog. Light Nodes nur einen zeitlichen Abschnitt der Datensätze vorhalten, um neue Transaktionen validieren zu können. In der generellen Definition wird von sog. Full Nodes ausgegangen in denen stets alle Datensätze vorgehalten werden.} des Datenbestands vorgehalten. Diese dezentrale Struktur bedeutet, dass ein \textit{Blockchain} Netzwerk nicht unter der Kontrolle oder Regulierung einer einzelnen Entität steht. Jeder Teilnehmer kann eigenständig im Netzwerk agieren und es ist kein Zwischenhändler nötig \citep{Drescher2017}.

Wird von einem der Teilnehmer eine Transaktion ausgelöst, wird diese nicht durch einen Intermediär sondern durch das Netzwerk erfasst und verarbeitet (Abbildung \ref{fig:change-in-transaction-model-blockchain}). Ein neuer \textit{Block} wird erschaffen und validiert wie es durch das Konsensprotokoll festgelegt wird.

\begin{figure}[h!]
	\centering
	\includegraphics[width=1.0\linewidth]{pictures/change-in-transaction-model-blockchain}
	\caption[Transaktionsmodell Blockchain]{Transaktionsmodell Blockchain}
	\label{fig:change-in-transaction-model-blockchain}
\end{figure}

Dabei können solche \textit{Blockchain} Systeme ziemlich unterschiedlich ausgeprägt sein. Unterscheiden lassen sich diese Systeme zb. an der Art des Zugriffs, also wer darf Transaktionen lesen, wer darf sie schreiben. Außerdem kann der Mechanismus zur Konsensfindung je System anders sein.

% \citep{Abeyratne2016}
% \citep{Casino2019}

% \citep{Platzer2014}
% \citep{Narayanan2016}
% \citep{Burgwinkel2016}

% \citep{Gayvoronskaya2017}
% \citep{Vigna2017}
% \citep{JPMorgan2018}
% \citep{Buhl2017}
% \citep{Maull2017}
% \citep{Technik2018}
% \citep{Mitschele2018}
% \citep{Neugebauer2018}
% \citep{Min2018}


\subsubsection{Begriffliche Abgrenzung}

Die am häufigsten verwendeten Begriffe werden im Folgenden anhand eines Schichtenmodells (Abbildung \ref{fig:layer-model-blockchain}) erklärt und voneinander abgegrenzt.

\begin{figure}[h!]
	\centering
	\includegraphics[width=1.0\linewidth]{pictures/layer-model-blockchain}
	\caption[Schichtenmodell \textit{Blockchain} Begriffe]{Schichtenmodell \textit{Blockchain} Begriffe QUELLE}
	\label{fig:layer-model-blockchain}
\end{figure}

\paragraph{Distributed Ledger}$~~$\\
Der \textit{Distributed Ledger} bildet die Basis des Schichtenmodells. Er ist im Grunde genommen ein klassisches Betandsbuch, das über einen Mechanismus verfügt, es auf alle teilnehmenden Parteien zu verteilen. \textit{Distributed Ledger} existieren bereits seit längerer Zeit und sind meist auf der technischen Basis einer verteilten Datenbank mit einer Logik auf Programm- oder Datenbankseite versehen, die aus der reinen Datenbank ein Bestandsbuch macht.

Distributed Ledger Technologie wird zunehmend synonym zum bisherigen Gebrauch von \textit{Blockchain} genutzt, um die Entwicklungen nach dem Bitcoin und den Kryptowährungen von eben diesen begrifflich abzugrenzen.

\paragraph{Blockchain-Technologie}$~~$\\
Die \textit{Blockchain} ist eine Form, einen \textit{Distributed Ledger} zu organisieren und zu implementieren. Auf die technische Implementierung der \textit{Blockchain} wird in den folgenden Kapiteln näher eingegangen; zur Begriffsbestimmung seien hier die grundlegenden Eigenschaften aufgezählt, die der \textit{Blockchain} in den letzten Jahren die steigende Aufmerksamkeit ermöglich haben:

\begin{itemize}
  \item Dezentralisiert
  \item Peer-to-Peer
  \item Transparenz und Anonymität
  \item Vertrauen
\end{itemize}

Blockchain gehört zu den bekanntesten Distributed-Ledger-Technologien. Aus diesem Grund wird die Bezeichnung Blockchain-Technologie in dieser Arbeit synonym für Distributed-Ledger-Technologien benutzt. Auf die technischen Eigenschaften von weiteren Ausprägungen der Distributed-Ledger-Technologien wird in dieser Arbeit daher nicht eingegangen.

\paragraph{Kryptowährungen}$~~$\\
Mit der \textit{Blockchain} als Basistechnologie lassen sich darauf aufbauende komplexe Systeme, wie z.B. Währungen abbilden. Wie in Kapitel \ref{blockchain-definition} erwähnt wurde die Blockchain-Technologie als erstes im Zusammenhang mit einer Kryptowährungen, dem Bitcoin, beschrieben. Die \textit{Blockchain} ist somit ein Nebenprodukt einer technischen Plattform, die eine kryptographische Währung erschuf und gleichzeitig ein System implementierte, um diese Währung zu nutzen und zu handeln.

Neben dem Bitcoin existiert eine Reihe weiterer Kryptowährungen, die sich zum Teil der dem Bitcoin zugrunde liegenden öffentlichen \textit{Blockchain} bedienen. Genannt seien hier z.B. Litecoin oder Dogecoin. Es existieren darüber hinaus Kryptowährungen, die eigene Blockchains zur Basis haben - zum Teil auf einer komplett eigenen technischen Implementierung. Vertreter hierfür sind z.B. Ethereum, Ripple oder Iota \citep[siehe auch][]{Buterin2014, carVertical, JPMorgan2018}.

\paragraph{Bitcoin}$~~$\\
Der Bitcoin ist die Kryptowährung, die auf der ursprünglichen \textit{Blockchain} gehandelt wird. Im Rahmen dieser Arbeit wird der Bitcoin und andere Kryptowährungen nicht weiter betrachtet.

\subsubsection{Arten von \textit{Blockchain}} \label{Arten-von-Blockchain}

Bei der Auswahl der Art einer \textit{Blockchain} trifft man auf zwei Widersprüche.

\begin{itemize}
  \item Transparenz vs. Vertraulichkeit
  \item Sicherheit vs. Geschwindigkeit
\end{itemize}

\paragraph{Transparenz vs. Vertraulichkeit}$~~$\\
Verwendet man eine \textit{Blockchain} werden Besitzverhältnisse durch die Transaktionshistorie ermittelt. Dabei lässt sie eine \textit{Blockchain} mit einem öffentlichen Register vergleichen. Im Sinne der Übertragung von Eigentum sind Offenheit und Transparenz zwei wesentliche Eigenschaften der Blockchain. Durch diese Offenheit ist jeder Teilnehmer in der Lage alle Transaktionen einzusehen und auf Manipulationen zu prüfen.

Dieses Vorgehen steht Gegensatz zur Vertraulichkeit die in bestimmten Bereichen unabdingbar ist. Durch Vertraulichkeit werden Informationen wie die Transaktionsdaten oder deren Details (beteiligte Konten oder transferierte Menge) vor unbefugter Einsicht geschützt. Hierdurch entsteht der Widerspruch zwischen Transparenz auf der einen Seite und Anforderungen an die Vertraulichkeit auf der anderen Seite \citep{Drescher2017}.

\paragraph{Sicherheit vs. Geschwindigkeit}$~~$\\
Die Datenstruktur einer \textit{Blockchain} sichert die Transaktionshistorie vor Manipulationen und Fälschungen. Jeder neue \textit{Block} der in der \textit{Blockchain} gespeichert werden soll muss vom Netzwerk durch das Lösen einer kryptographischen Aufgabe erzeugt und der Datenstruktur hinzugefügt werden. Dadurch ist es ziemlich aufwendig die Transaktionshistorie nachträglich zu manipulieren oder zu fälschen. Durch diesen Sicherheitsmechanismus sinkt die Geschwindigkeit mit der ein \textit{Blockchain} Netzwerk neue Transaktionen verarbeiten kann. Moderne Applikationen erfordern Geschwindigkeit und Skalierbarkeit was im direkten Kontrast zum erwähnten Sicherheitskonzept einer \textit{Blockchain} steht \citep{Drescher2017}.

\paragraph{Ursachen der Konflikte}$~~$\\
Zwei grundlegende Operationen eines \textit{Blockchain} Netzwerks sind Ursache für die beiden beschriebenen Widersprüche - Schreiben und Lesen von Transaktionsdaten. Der Konflikt zwischen Transparenz und Vertraulichkeit ist auf die Lese-Operationen einer \textit{Blockchain} zurückzuführen. Je offener die Leseberechtigungen einer \textit{Blockchain} sind, desto höher ist die Transparenz und desto niedriger ist die Vertraulichkeit der Transaktionsdaten. Die Schreib-Operationen sind für den Widerspruch zwischen Sicherheit und Geschwindigkeit verantwortlich. Je restriktiver die Berechtigungen zum Schreiben innerhalb des \textit{Blockchain} Netzwerks sind, desto höher ist die Geschwindigkeit mit der Transaktionen verarbeitet werden können. In Tabelle \ref{tab:technical-restricts-blockchain} werden die technischen Beschränkungen, der Widerspruch und die Operation innerhalb der \textit{Blockchain} zusammengefasst \citep{Drescher2017}.

\begin{table}[htb]\centering
  \begin{tabularx}{\textwidth}{XXX}
    \toprule
    \textbf{Beschränkung} &\textbf{Widerspruch} &\textbf{Blockchain Operation}\\ \midrule
    Keine Vertraulichkeit & Transparenz vs. Vertraulichkeit & Transaktionshistorie lesen\\ \addlinespace
    Skalierbarkeit & Sicherheit vs. Geschwindigkeit & Transaktionen schreiben\\
    \bottomrule
  \end{tabularx}
  \caption{Technische Beschränkungen der \textit{Blockchain} und ihre Ursachen}
  \label{tab:technical-restricts-blockchain}
\end{table}

\paragraph{Public vs. Private}$~~$\\
Betrachtet man die Berechtigungen zum Lesen innerhalb eines \textit{Blockchain} Netzwerks in der einfachsten Form muss das System zwischen Transparenz und Vertraulichkeit entscheiden. Entweder es werden allen Teilnehmern Leseberechtigungen zugeteilt oder nur einer ausgewählten Gruppe von Teilnehmern. Anhand des Kriterium, welcher Teilnehmer im Netzwerk neue Transaktionen erstellen und die Historie lesen kann, lässt sich eine \textit{Blockchain} als öffentliche oder private \textit{Blockchain} charakterisieren \citep{Drescher2017}.

\paragraph{Permissioned vs. Permissionless}$~~$\\
Die Schreibrechte bestimmen für ein \textit{Blockchain} Netzwerk den Grad der Skalierbarkeit. Werden Schreibrechte in ihrer einfachsten Form zugeteilt und alle Teilnehmer sind berechtigt Schreib-Operationen auszuführen, erhöht sich der Arbeitsaufwand je Teilnehmer der zur Berechnung nötigt wird. Dies ist für die Sicherheit des Netzwerk positiv, wirkt sich aber negativ auf die Geschwindigkeit aus. Durch die Geschwindigkeit wird das Netzwerk in der Skalierbarkeit beschränkt. Teilt man hingegen nur einer Gruppe von Teilnehmern Schreibrechte zu, ist der Arbeitsaufwand im Vergleich niedrig. Hierdurch kann das Netzwerk Transaktionen vergleichsweise schnell verarbeiten und ist dadurch selbst skalierbarer \citep{Drescher2017}.

\begin{figure}[h!]
	\centering
	\includegraphics[width=1.0\linewidth]{pictures/dlt-type-matrix}
	\caption[Distributed Ledger Type Matrix]{Distributed Ledger Type Matrix ERSETZEN}
	\label{fig:dlt-type-matrix}
\end{figure}

\subsubsection{Technologischer Aufbau}
tl;dr Architektur Modell Abbildung erläutern. In Details einleiten

\paragraph{Peer-to-Peer Netzwerke}$~~$\\
Grundlegende Technik dahinter erklären in 1-2 Sätzen. Bekanntester und größter Einsatzzweck Torrent Netzwerk.

\paragraph{Signierte Transaktionen durch Public-Key-Infrastruktur}$~~$\\
Was ist Public-Key-Infrastruktur?
Beispiel Wallets und Asset X das gehandelt wird.

\paragraph{Kryptographisches Hashing}$~~$\\
Unterschied zwischen Hashing und krypt. Hashing aufzeigen. Matheformel :D
Evtl. die kleine Bilderreihe mit dem sich verändernden Hash Wert.
\citep{Diffie1976}

\paragraph{Konsensusprotokolle}$~~$\\
Byzantine Fault Tolerance erklären als Einstieg

Aufteilen nach Häufigkeit in der jeweiligen \textit{Blockchain} Art aus \ref{Arten-von-Blockchain}.
Ausschluss weiterer Betrachtung einiger Algorithmen für den weiteren Verlauf der Arbeit.


\subparagraph{Proof-of-X}$~~$\\
\begin{figure}[h!]
	\centering
	\includegraphics[width=1.0\linewidth]{pictures/placeholder_half_page}
	\caption[Placeholder Half Page]{Placeholder Half Page}
	\label{fig:placeholder_half_page}
\end{figure}

\subparagraph{Redundant Byzantine Fault Tolerance \& Plenum}$~~$\\
\begin{figure}[h!]
	\centering
	\includegraphics[width=1.0\linewidth]{pictures/placeholder_half_page}
	\caption[Placeholder Half Page]{Placeholder Half Page}
	\label{fig:placeholder_half_page}
\end{figure}


\newpage
