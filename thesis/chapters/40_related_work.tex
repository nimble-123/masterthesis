\section{Verwandte Arbeiten} \label{sec:related-work}
In diesem Kapitel soll ein kurzer Überblick über die bereits vorhandenen Lösungen im Bereich der Lebensmittelsicherheit und Supply Chain gegeben werden. Im Fokus der Betrachtung liegt die jeweils verwendete Blockchain-Technologie für den spezifischen Use-Case, sowie die tatsächliche Umsetzung und der Stand der Lösung.

\subsection{Thunfisch Traceability}
Der World Wildlife Fund (WWF) in Australien, Fidschi und Neuseeland hat in Zusammenarbeit mit dem US-amerikanischen Technologie-Innovator ConsenSys, dem Technologie-Implementierer TraSeable und dem Thunfischfang- und -verarbeitungs\-unternehmen Sea Quest Fiji Ltd., ein Pilotprojekt in der Thunfischindustrie der Pazifikinseln gestartet, das mit Hilfe der Blockkettentechnologie den Weg des Thuns vom \glqq Köder auf den Teller\grqq{} verfolgen wird. Ziel ist es, dazu beizutragen, illegale, nicht gemeldete und unregulierte Fischerei und Menschenrechtsverletzungen in der Thunfischindustrie zu stoppen. Dazu gehören Berichte über Korruption, illegalen Handel und menschliche Sklaverei auf Thunfischfängern.

Das WWF-Pilotprojekt wird eine Kombination aus RFID-Tags (Radio Frequency Identification), QR-Code-Tags (Quick Response) und Lesegeräten verwenden, um Informationen über die Reise eines Thunfisches an verschiedenen Punkten der Lieferkette zu sammeln. Während dieser Technologieeinsatz für das Supply-Chain-Tracking nicht neu ist, ist der spannende Teil, dass die gesammelten Informationen dann mit Hilfe der Blockchain-Technologie aufgezeichnet werden. Die Ortung beginnt, sobald der Thunfisch gefangen wird. Sobald ein Fisch gelandet ist, wird er mit einem wiederverwendbaren RFID-Tag auf dem Schiff befestigt. Geräte, die auf dem Schiff, am Dock und in der Verarbeitungsfabrik angebracht sind, erkennen dann die Tags und laden automatisch Informationen in die Blockkette hoch. 

Nach der Verarbeitung des Fisches wird der wiederverwendbare RFID-Tag gegen einen kostengünstigeren QR-Code ausgetauscht, der an der Produktverpackung angebracht wird. Der eindeutige QR-Code wird mit dem Blockchain-Datensatz verknüpft, der dem jeweiligen Fisch und seinem ursprünglichen RFID-Tag zugeordnet ist. Der QR-Code wird verwendet, um den Rest der Reise des Fisches zum Verbraucher zu verfolgen. Im Moment ist die Verknüpfung von Tags nicht schwierig, da sich das Projekt auf den gesamten Export konzentriert - also den gesamten frischen Fisch abzüglich Kopf, Kiemen und Eingeweiden. Etwas komplizierter wird es, wenn der Fisch in Lenden, Steaks, Würfel und Dosen zerlegt wird, aber das Projektteam ist nun in der Lage, die QR-Code-Tags auf den Verpackungen des verarbeiteten Fisches mit dem Datensatz des Originalfisches auf der Blockkette zu verknüpfen. Auch wenn es möglich sein könnte, RFID-Tags während des gesamten Prozesses zu verwenden, könnten die Kosten dieser Tags kleineren Unternehmen in der Fischwirtschaft die Teilnahme an dem System verbieten, wenn es sich ausweitet. Es besteht auch das Potenzial, in Zukunft mit Nahfeldkommunikationsgeräten (NFC) die Fische bis zum Verbraucher zu verfolgen.

\citep{Visser2017}
\citep{McEntire2019}

\subsection{Halal Food Chain}
\citep{Tan2018}
\citep{Zailani2010}
\citep{Rejeb2018}
\textcolor{red}{Lorem ipsum dolor sit amet, consetetur sadipscing elitr, sed diam nonumy eirmod tempor invidunt ut labore et dolore magna aliquyam erat, sed diam voluptua. At vero eos et accusam et justo duo dolores et ea rebum. Stet clita kasd gubergren, no sea takimata sanctus est Lorem ipsum dolor sit amet. Lorem ipsum dolor sit amet, consetetur sadipscing elitr, sed diam nonumy eirmod tempor invidunt ut labore et dolore magna aliquyam erat, sed diam voluptua. At vero eos et accusam et justo duo dolores et ea rebum. Stet clita kasd gubergren, no sea takimata sanctus est Lorem ipsum dolor sit amet. Lorem ipsum dolor sit amet, consetetur sadipscing elitr, sed diam nonumy eirmod tempor invidunt ut labore et dolore magna aliquyam erat, sed diam voluptua. At vero eos et accusam et justo duo dolores et ea rebum. Stet clita kasd gubergren, no sea takimata sanctus est Lorem ipsum dolor sit amet. Lorem ipsum dolor sit amet, consetetur sadipscing elitr, sed diam nonumy eirmod tempor invidunt ut labore et dolore magna aliquyam erat, sed diam voluptua. At vero eos et accusam et justo duo dolores et ea rebum. Stet clita kasd gubergren, no sea takimata sanctus est Lorem ipsum dolor sit amet. Lorem ipsum dolor sit amet, consetetur sadipscing elitr, sed diam nonumy eirmod tempor invidunt ut labore et dolore magna aliquyam erat, sed diam voluptua. At vero eos et accusam et justo duo dolores et ea rebum. Stet clita kasd gubergren, no sea takimata sanctus est Lorem ipsum dolor sit amet. Lorem ipsum dolor sit amet, consetetur sadipscing elitr, sed diam nonumy eirmod tempor invidunt ut labore et dolore magna aliquyam erat, sed diam voluptua. At vero eos et accusam et justo duo dolores et ea rebum. Stet clita kasd gubergren, no sea takimata sanctus est Lorem ipsum dolor sit amet.}

\newpage
