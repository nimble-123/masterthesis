\section{Verwandte Arbeiten} \label{sec:related-work}
In diesem Kapitel soll ein kurzer Überblick über zwei vorhandene Lösungen im Bereich der Lebensmittelsicherheit und Supply Chain gegeben werden. Im Fokus der Betrachtung liegt die jeweils verwendete \textit{Blockchain-Technologie} für den spezifischen \textit{Use-Case}, sowie die tatsächliche Umsetzung und der Stand der Lösung.

\subsection{Thunfisch Traceability}
Der \ac{wwf} in Australien, Fidschi und Neuseeland hat in Zusammenarbeit mit dem US-amerikanischen Technologie-Innovator ConsenSys, dem Technologie-Implementierer TraSeable und dem Thunfischfang- und -verarbeitungs\-unternehmen Sea Quest Fiji Ltd., ein Pilotprojekt in der Thunfischindustrie der Pazifikinseln gestartet, das mit Hilfe der \textit{Blockchain-Technologie} den Weg des Thunfisches vom \glqq Köder auf den Teller\grqq{} verfolgen wird. Ziel ist es, dazu beizutragen, illegale, nicht gemeldete und unregulierte Fischerei und Menschenrechtsverletzungen in der Thunfischindustrie zu stoppen. Dazu gehören Berichte über Korruption, illegalen Handel und menschliche Sklaverei auf Thunfischfängern.

Das \ac{wwf}-Pilotprojekt wird eine Kombination aus \textit{\ac{rfid}-Tags}, \textit{\ac{qr}-Code-Tags} und Lesegeräten verwenden, um Informationen über die Reise eines Thunfisches an verschiedenen Punkten der Lieferkette zu sammeln. Während dieser Technologieeinsatz für das \textit{Supply-Chain-Tracking} nicht neu ist, ist der innovative Teil, dass die gesammelten Informationen dann mit Hilfe der \textit{Blockchain-Technologie} aufgezeichnet werden. Die Ortung beginnt, sobald der Thunfisch gefangen wird. Sobald ein Fisch gelandet ist, wird er mit einem wiederverwendbaren \textit{\ac{rfid}-Tag} auf dem Schiff befestigt. Geräte, die auf dem Schiff, am Dock und in der Verarbeitungsfabrik angebracht sind, erkennen dann die \textit{Tags} und laden automatisch Informationen in die \textit{Blockchain} hoch. 

Nach der Verarbeitung des Fisches wird der wiederverwendbare \textit{\ac{rfid}-Tag} gegen einen kostengünstigeren \textit{\ac{qr}-Code} ausgetauscht, der an der Produktverpackung angebracht wird. Der eindeutige \textit{\ac{qr}-Code} wird mit dem \textit{Blockchain}-Datensatz verknüpft, der dem jeweiligen Fisch und seinem ursprünglichen \textit{\ac{rfid}-Tag} zugeordnet ist. Der \textit{\ac{qr}-Code} wird verwendet, um den Rest der Reise des Fisches zum Verbraucher zu verfolgen. Im Moment ist die Verknüpfung von \textit{Tags} nicht schwierig, da sich das Projekt auf den gesamten Export konzentriert - also den gesamten frischen Fisch abzüglich Kopf, Kiemen und Eingeweiden. Etwas komplizierter wird es, wenn der Fisch in Lenden, Steaks, Würfel und Dosen zerlegt wird, aber das Projektteam ist nun in der Lage, die \textit{\ac{qr}-Code-Tags} auf den Verpackungen des verarbeiteten Fisches mit dem Datensatz des Originalfisches auf der \textit{Blockchain} zu verknüpfen. Auch wenn es möglich sein könnte, \textit{\ac{rfid}-Tags} während des gesamten Prozesses zu verwenden, könnten die Kosten dieser \textit{Tags} kleineren Unternehmen in der Fischwirtschaft die Teilnahme an dem System verbieten, wenn es sich ausweitet. Es besteht auch das Potenzial, in Zukunft mit \textit{\ac{nfc}} die Fische bis zum Verbraucher zu verfolgen \citep{Visser2017, McEntire2019}.

\subsection{Halal Food Chain}
Da Lebensmittel zwischen den verschiedenen Akteuren der Lieferkette verstreut sind, wächst die Sorge um die Gewährleistung der Lebensmittelsicherheit durch die Einführung vieler Internet- und Sachtechnologien. Neben der Unsicherheit ist die Möglichkeit, dass Halal-Lebensmittel nicht Halal sind, aufgrund der Fahrstrecke, die viele Handhabungspunkte einschließt, und des anhaltenden Risikos einer Kreuzkontamination mit Nicht-Halal-Materialien größer. Um die Fragen im Zusammenhang mit der strengen Einhaltung des Scharia-Rechts durch Halal-Produkte zu klären, gibt es einige Vorveröffentlichungen in Studien über die Rückverfolgbarkeit von Halal-Fleischprodukten \citep{Mohammed2016}. Als Lösung dafür schlug \citet{Mohamad2016} eine Methode vor, um zu bestimmen, ob das Geflügel nach islamischer Art und Weise unter Verwendung einer Untersuchung der Fleischfarbe geschlachtet wird. \citet{Junaini2008} beschreiben eine mobile Unterstützungsanwendung für Muslime zur Identifizierung des Halal-Status. \citet{Kassim2012} führten ein System ein, um die Informationen von Produkten zu verifizieren und zu erkennen und damit ihren Halal-Status in Echtzeit von einem Echtzeit-Zugriff auf ihre Datenbank zu bestätigen. \citet{SitiSarahMohdBahrudin2011} schlugen eine umfassende und geeignete Tracking \& Tracing-Technologie mit \textit{\ac{rfid}} vor, um die Integrität des Halal-Produkts aufrechtzuerhalten und die gesamte Lieferkette des Halal-Produktprozesses zu unterstützen. \citet{Tan2012} fanden heraus, dass Technologien wie \textit{\ac{tms}}, \textit{\ac{wms}}, \textit{\ac{edi}} und \textit{\ac{gps}} bei Halal-Logistikdienstleistern weit verbreitet sind. Außerdem betonte \citet{Tan2012} die Kompatibilität der \textit{Tracking \& Tracing}-Eigenschaft von \textit{\ac{rfid}} mit der Halal-Transportrichtlinie. \citet{Mohammed2016} stellen einen Rahmen für die Entwicklung eines \textit{\ac{rfid}}-fähigen \textit{\ac{hmsc}}-Netzwerks zur Verbesserung der Rückverfolgbarkeit der Halal-Fleischintegrität in der gesamten Lieferkette vor. Alle zuvor genannten Forschungsarbeiten sind die Idee der Verwendung eines zentralen Systems, das letztlich der einzig denkbare Weg war, um Informationstransparenz entlang der Lieferketten zu erreichen \citep{Tian2017}. Es gibt jedoch nicht genügend Beweise für die Richtigkeit und Vertrauenswürdigkeit der gemeinsamen Informationen im Rückverfolgbarkeitssystem und zwischen den Akteuren der Halal-Fleischlieferkette. Dies führt zu einem undurchsichtigen System, Informationsasymmetrie und vielen anderen Problemen. Das \textit{\ac{rfid}}-fähige Rückverfolgbarkeitssystem reicht nicht aus, um die Halal-Integrität von Fleisch zu gewährleisten. Das manuelle Abrufen und Speichern von Informationen in der zentralen Datenbank bringt viele Möglichkeiten der Irreführung und Verfälschung mit sich. In ähnlicher Weise ist es problematisch, sicherzustellen, dass die so genannten Halal-Fleischprodukte den islamischen Ernährungsvorschriften entsprechen und frei von irreführenden Herkunftsgeschichten sind.

\newpage
