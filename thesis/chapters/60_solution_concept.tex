\section{Lösungskonzept}
Das Kapitel Lösungskonzept soll aufzeigen mit welcher konkreten Ausprägung der Blockchain Techologie der gewählte Use-Case realisiert werden soll. Dazu wird im ersten Schritt eine SWOT-Analyse zur Blockchain Technologie allgemein durchgeführt und die Ergebnise beschrieben. In Schritt zwei kommt eine Nutzwertanalyse zum Einsatz anhand welcher ermittelt wird welche Ausprägung der Technologie sich zur Umsetzung bestmöglichst eignet.

\subsection{SWOT-Analyse der \textit{Blockchain-Technologie}}
Durch die Vielzahl an unterschiedlichen Use-Cases die mittels der Blockchain Technologie umgesetzt werden ist es nötig für den spezifischen Use-Case der Chargenrückverfolgung die Technologie einer SWOT-Analyse zu unterziehen. Hierdurch wird gewährleistet, dass die Technologie für den Use-Case überhaupt geeignet ist. Im folgenden werden daher die Stärken und Schwächen gegenübergestellt, sowie die dadurch möglichen Chancen und Risiken diskutiert.

\subsubsection{Stärken}
\paragraph{Manipulationsicherheit}
Eine der Schlüsselstärken der Technologie ist, dass sie durch die Art und Weise wie Transaktionen gespeichert und verknüpft werden eine Manipulation von Datensätzen direkt erkennbar macht.

\paragraph{Schutz der Privatsphäre}
Durch eine Implementierung eines Berechtigungskonzepts können Teilnehmer des Netzwerks eigenständig definieren wer auf die Daten zugreifen kann, für welchen Zweck und für welchen Zeitraum. Diese Regeln werden in Smart Contracts programmatisch abgebildet und bei jeder Ausführung geprüft.

\paragraph{Effizienzsteigerung}
Zusätzlich zur Manipulationsicherheit und dem Schutz der Privatsphäre bietet die Blockchain Technologie die Möglichkeit der Effizienzsteigerung für Geschäftsprozesse. Durch den Einsatz von Kryptographie können zwei Parteien vertrauensvoll miteinander interagieren. Eine gesonderte Prüfung der Transaktionen entfällt hierbei, da sie durch Smart Contracts bereits geprüft wurde. Hierdurch entsteht ein Einsparungspotential bzw. eine Effizienzsteigerung.

\subsubsection{Schwächen}
\paragraph{Mangelnde Kontrolle}
In der Theorie sind Blockchain Lösungen dezentralisiert und selbstverwaltend in der Praxis zeigt sich jedoch, dass der Betrieb eines solchen Systems maßgeblich unter der Kontrolle einer Gruppe von Entwicklern bzw. einer eigens dafür gegründeten Organisation steht.

\paragraph{Fehlender Kontext}


\paragraph{Verletzung des Datenschutzes}

\subsubsection{Chancen}
\paragraph{Neue Geschäftsmodelle}

\paragraph{Verbesserung des Vertrauens in Transaktionen}

\subsubsection{Risiken}

\subsection{Nutzwertanalyse}

\subsection{???}

\subsection{Zusammenfassung Lösungskonzept}

\newpage
