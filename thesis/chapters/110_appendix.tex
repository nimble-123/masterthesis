\appendix
%\begin{appendix}

%\addcontentsline{toc}{section}{Anhang}
%\section*{Anhang}
\section{Anhang}
\subsection*{Funktionale Anforderungen}
%\subsection*{Funktionale Anforderungen}
\begin{table}[H]
    \begin{tabularx}{\textwidth}{@{}lXp{2cm}@{}}
        \toprule
        ID                & Anforderung & Quelle \\
        \midrule
        \textbf{A1.1}              & Das Gesamtsystem muss fähig sein den Lebenszyklus eines Tieres vom Erzeuger bis zum Lebensmitteleinzelhandel abzubilden.                    & \textit{Wissensch. Kontext}                \\ \addlinespace
        \multicolumn{1}{r}{A1.1.1} & Das Gesamtsystem muss fähig sein Tiere anzulegen/registrieren.                     &                 \\ \addlinespace
        \multicolumn{1}{r}{A1.1.2} & Das Gesamtsystem muss fähig sein Tiere und Chargen einander zuzuordnen.                     &                 \\ \addlinespace
        \multicolumn{1}{r}{A1.1.3} & Das Gesamtsystem muss fähig sein Tiere zwischen Teilnehmern zu transferieren im Sinne eines Eigentumswechsel.                     &                 \\
        \textbf{A1.2}              & Das Gesamtsystem muss eine generische Schnittstelle zur Kommunikation mit dem Ledger anbieten.                     & \textit{Partner}                \\ \addlinespace
        \textbf{A1.3}              & Das Gesamtsystem muss fähig sein Transaktionsdaten manipulationssicher speichern zu können.                     & \textit{Partner}                \\ \addlinespace
        \textbf{A1.4}              & Das Gesamtsystem muss fähig sein den Lebenszyklus einer Charge abzubilden.                     & \textit{Partner}                \\ \addlinespace
        \multicolumn{1}{r}{A1.4.1} & Das Gesamtsystem muss fähig sein Chargen anzulegen.                     &                 \\ \addlinespace
        \multicolumn{1}{r}{A1.4.2} & Das Gesamtsystem muss fähig sein Chargen und Tiere einander zuzuordnen.                     &                 \\ \addlinespace
        \bottomrule
    \end{tabularx}
    \caption{Funktionale Anforderungen}
    \label{tab:functional-requirements}
\end{table}

\subsection{Rahmenbedingungen}
\begin{table}[H]
    \begin{tabularx}{\textwidth}{@{}lXp{2cm}@{}}
        \toprule
        ID                & Anforderung & Quelle \\
        \midrule
        \textbf{A2.1}              & Der Prototyp muss mit der Hyperledger Fabric Blockchain Technologie konzipiert und implementiert werden.                     & \textit{Partner}                \\ \addlinespace
        \textbf{A2.2}              & Der Prototyp bildet die Teilnehmer der Wirtschöpfungskette vom Erzeuger bis zum Lebensmitteleinzelhandel ab.                     & \textit{Partner}                \\ \addlinespace
        \textbf{A2.3}              & Der Prototyp fokussiert sich bei der Transaktionsabwicklung auf die Tierart Schwein. (Verminderte Komplexität)                     & \textit{Partner}                \\ \addlinespace
        \textbf{A2.4}\phantomsection\label{req:A2.4}              & Das Gesamtsystem muss in einer abgeschlossenen Umgebung gehosted und vor pseudonymen Zugriff geschützt sein.                     & \textit{Partner}                \\ \addlinespace
        \bottomrule
    \end{tabularx}
    \caption{Funktionale Anforderungen}
    \label{tab:functional-requirements}
\end{table}

\subsection{Qualitätsanforderungen}
\begin{table}[H]
    \begin{tabularx}{\textwidth}{@{}lXp{2cm}@{}}
        \toprule
        ID                & Anforderung & Quelle \\
        \midrule
        \textbf{A3.1}              & Die Architektur des Systems muss eine nachträgliche Erweiterung ermöglichen, um weitere Geschäftszweige abbilden zu können.                     & \textit{Wissensch. Kontext}                \\ \addlinespace
        \textbf{A3.2}              & \textcolor{red}{Die Architektur des Systems muss mindestens eine konstante Performance bei steigender Teilnehmerzahl.}                     & \textit{Partner}                \\ \addlinespace
        \textbf{A3.3}              & Das System muss auch bei Ausfall oder Komprimitierung eines oder mehrerer Teilnehmer konsistent und stabil weiter arbeiten.                    & \textit{Partner}                \\ \addlinespace
        \bottomrule
    \end{tabularx}
    \caption{Funktionale Anforderungen}
    \label{tab:functional-requirements}
\end{table}

\newpage
\subsection{Listings}
\subsubsection{Listing A}
\begin{lstlisting}[caption={Hyperledger Fabric Peer \textit{Dockerfile}},captionpos=b,language=Dockerfile,label=lst:dockerfile-hl-peer]
FROM golang:1.11.5

ENV DEBIAN_FRONTEND noninteractive
ENV FABRIC_ROOT=$GOPATH/src/github.com/hyperledger/fabric
ENV CHAINTOOL_RELEASE=1.1.2

# Architecture of the node
ENV ARCH=amd64
# version for the base images (baseos, baseimage, ccenv, etc.), used in core.yaml as BaseVersion
ENV BASEIMAGE_RELEASE=0.4.14
# BASE_VERSION is required in core.yaml for the runtime fabric-baseos
ENV BASE_VERSION=1.4.0
# version for the peer/orderer binaries, the community version tracks the hash value like 1.0.0-snapshot-51b7e85
# PROJECT_VERSION is required in core.yaml to build image for cc container
ENV PROJECT_VERSION=1.4.0
# generic golang cc builder environment (core.yaml): builder: $(DOCKER_NS)/fabric-ccenv:$(ARCH)-$(PROJECT_VERSION)
ENV DOCKER_NS=hyperledger
# for golang or car's baseos for cc runtime: $(BASE_DOCKER_NS)/fabric-baseos:$(ARCH)-$(BASEIMAGE_RELEASE)
ENV BASE_DOCKER_NS=hyperledger
ENV LD_FLAGS="-X github.com/hyperledger/fabric/common/metadata.Version=${BASE_VERSION} \
    -X github.com/hyperledger/fabric/common/metadata.BaseVersion=${BASEIMAGE_RELEASE} \
    -X github.com/hyperledger/fabric/common/metadata.BaseDockerLabel=org.hyperledger.fabric \
    -X github.com/hyperledger/fabric/common/metadata.DockerNamespace=hyperledger \
    -X github.com/hyperledger/fabric/common/metadata.BaseDockerNamespace=hyperledger \
    -X github.com/hyperledger/fabric/common/metadata.Experimental=true \
    -linkmode external -extldflags '-static -lpthread'"

# Peer config path
ENV FABRIC_CFG_PATH=/etc/hyperledger/fabric
RUN mkdir -p /var/hyperledger/db \
    /var/hyperledger/production \
    $GOPATH/src/github.com/hyperledger \
    $FABRIC_CFG_PATH \
    /chaincode/input \
    /chaincode/output

# Install development dependencies
RUN apt-get update \
        && apt-get install -y apt-utils python-dev \
        && apt-get install -y libsnappy-dev zlib1g-dev libbz2-dev libyaml-dev libltdl-dev libtool \
        && apt-get install -y python-pip \
        && apt-get install -y tree jq unzip\
        && rm -rf /var/cache/apt

# install chaintool
#RUN curl -L https://github.com/hyperledger/fabric-chaintool/releases/download/v0.10.3/chaintool > /usr/local/bin/chaintool \
RUN curl -fL https://nexus.hyperledger.org/content/repositories/releases/org/hyperledger/fabric/hyperledger-fabric/chaintool-${CHAINTOOL_RELEASE}/hyperledger-fabric-chaintool-${CHAINTOOL_RELEASE}.jar > /usr/local/bin/chaintool \
    && chmod a+x /usr/local/bin/chaintool

# install gotools
RUN go get github.com/golang/protobuf/protoc-gen-go \
    && go get github.com/maxbrunsfeld/counterfeiter \
    && go get github.com/axw/gocov/... \
    && go get github.com/AlekSi/gocov-xml \
    && go get golang.org/x/tools/cmd/goimports \
    && go get golang.org/x/lint/golint \
    && go get github.com/estesp/manifest-tool \
    && go get github.com/client9/misspell/cmd/misspell \
    && go get github.com/estesp/manifest-tool \
    && go get github.com/onsi/ginkgo/ginkgo

# Clone the Hyperledger Fabric code and cp sample config files
RUN cd $GOPATH/src/github.com/hyperledger \
    && git clone --single-branch -b release-1.4 --depth 1 http://gerrit.hyperledger.org/r/fabric \
    && cp $FABRIC_ROOT/devenv/limits.conf /etc/security/limits.conf \
    && cp -r $FABRIC_ROOT/sampleconfig/* $FABRIC_CFG_PATH/ \
    && cp $FABRIC_ROOT/examples/cluster/config/configtx.yaml $FABRIC_CFG_PATH/ \
    && cp $FABRIC_ROOT/examples/cluster/config/cryptogen.yaml $FABRIC_CFG_PATH/

# install configtxgen, cryptogen and configtxlator
RUN cd $FABRIC_ROOT/ \
    && go install -tags "experimental" -ldflags "${LD_FLAGS}" github.com/hyperledger/fabric/common/tools/configtxgen \
    && go install -tags "experimental" -ldflags "${LD_FLAGS}" github.com/hyperledger/fabric/common/tools/cryptogen \
    && go install -tags "experimental" -ldflags "${LD_FLAGS}" github.com/hyperledger/fabric/common/tools/configtxlator

# Install eventsclient
RUN cd $FABRIC_ROOT/examples/events/eventsclient \
    && go install \
    && go clean

# Install discover cmd
RUN CGO_CFLAGS=" " go install -tags "experimental" -ldflags "-X github.com/hyperledger/fabric/cmd/discover/metadata.Version=${BASE_VERSION}" github.com/hyperledger/fabric/cmd/discover

# The data and config dir, can map external one with -v
VOLUME /var/hyperledger
#VOLUME /etc/hyperledger/fabric

# temporarily fix the `go list` complain problem, which is required in chaincode packaging, see core/chaincode/platforms/golang/platform.go#GetDepoymentPayload
ENV GOROOT=/usr/local/go

WORKDIR $FABRIC_ROOT

# This is only a workaround for current hard-coded problem when using as fabric-baseimage.
RUN ln -s $GOPATH /opt/gopath
LABEL org.hyperledger.fabric.version=${PROJECT_VERSION} \
    org.hyperledger.fabric.base.version=${BASEIMAGE_RELEASE}
\end{lstlisting}

\subsubsection*{Listing B}
\begin{lstlisting}[caption={Hyperledger Fabric Network \textit{Connection Profile}},captionpos=b,language=json,label=lst:connection-profile]
    {
        "name": "hlfv1",
        "x-type": "hlfv1",
        "x-commitTimeout": 300,
        "version": "1.0.0",
        "client": {
            "organization": "Org1",
            "connection": {
                "timeout": {
                    "peer": {
                        "endorser": "300",
                        "eventHub": "300",
                        "eventReg": "300"
                    },
                    "orderer": "300"
                }
            }
        },
        "channels": {
            "composerchannel": {
                "orderers": [
                    "orderer.example.com"
                ],
                "peers": {
                    "peer0.org1.example.com": {
                        "endorsingPeer": true,
                        "chaincodeQuery": true,
                        "ledgerQuery": true,
                        "eventSource": true
                    }
                }
            }
        },
        "organizations": {
            "Org1": {
                "mspid": "Org1MSP",
                "peers": [
                    "peer0.org1.example.com"
                ],
                "certificateAuthorities": [
                    "ca.org1.example.com"
                ]
            }
        },
        "orderers": {
            "orderer.example.com": {
                "url": "grpc://orderer.example.com:7050"
            }
        },
        "peers": {
            "peer0.org1.example.com": {
                "url": "grpc://peer0.org1.example.com:7051"
            }
        },
        "certificateAuthorities": {
            "ca.org1.example.com": {
                "url": "http://ca.org1.example.com:7054",
                "caName": "ca.org1.example.com"
            }
        }
    }
\end{lstlisting}

\newpage
% \addcontentsline{toc}{section}{Literaturverzeichnis}
\section{Literaturverzeichnis}
\renewcommand{\refname}{B LITERATURVERZEICHNIS}
\bibliographystyle{apalike}
%\bibliography{thesis} % Point to BibTeX literature file e.g. literatur.bib
{\def\section*#1{}\bibliography{thesis}}

% \end{appendix}
\newpage
