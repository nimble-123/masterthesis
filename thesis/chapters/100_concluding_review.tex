\section{Abschlussbetrachtung} \label{sec:concluding-review}

\subsection{Zusammenfassung}

In dieser Masterarbeit wurde analysiert, wie sich eine Chargenrückverfolgung mittels der Blockchain Technologie realisieren lässt. Dazu wurden die in Abschnitt \ref{Problemstellung} gestellten Forschungsfragen anhand der Design Science Methode nach \citet{Hevner2007} bearbeitet. Die einzelnen Teilfragen wurden in den Kapiteln \ref{sec:solution-concept} und \ref{sec:system-design} mit einer der Fragestellung passenden Methodik näher betrachtet. Die Forschungsfrage FF1.1 sowie FF1.2 sind über die Grundlagenkapitel abgedeckt worden. In diesen Kapiteln wurde detailliert beschrieben, welche Anforderungen und Daten zur Realisierung einer Chargenrückverfolgung in der Fleischwarenindustrie von nöten sind. Neben einer ausführlichen Beschreibung der Wertschöpfungskette im fleischverarbeitenden Gewerbe wurde die Blockchain Technologie und ihre Ausprägungen behandelt. Forschungsfrage FF1.3 wurde mittels einer SWOT-Analyse mit anschließender Nutzwertanalyse entgegen getreten. Aus den Ergebnissen der Analyse wurde dann im Kapitel \ref{sec:system-design} ein entsprechendes System Design abgeleitet, welches für den Anwendungsfall passend ist und die in FF1.1 ermittelten Anforderungen erfüllt. Nach dem Systementwurf folgte die prototypische Implementierung des zuvor modellierten Systems auf Basis der Hyperledger Fabric Blockchain in Kombination mit dem Hyperledger Composer Framework zur Smart Contract generierung. Evaluiert wurde der Prototyp anhand eines Experteninterviews. Die Befragung einer Person mit direktem Bezug zu den behandelten Geschäftsprozessen sowie der nötigen Kompetenz bezüglich neuartiger Technologien wie Blockchain und \ac{iot} stellt eine für diese Arbeit ausreichend gesicherte Evaluation der Ergebnisse aus Systementwurf und dem resultierenden Prototyp dar.

\subsection{Reflexion}

Während der Anforderungsanalyse hat sich gezeigt, das die Blockchain Technologie in den Fachabteilungen des Praxispartners zwar bekannt war, ihre möglichen Einsatzwecke jedoch noch vollkommen unklar sind. Blockchain wurde stets mit der Kryptowährung Bitcoin assoziiert. So war es schwierig die Anforderung entsprechend spezifisch und nicht zu allgemein zu erheben ohne das wichtige Aspekte des zu modellierenden System außer acht gelassen werden. Auf Grund der Komplexität im realen Umfeld der Chargenrückverfolgung wurde der aufgenommen Prozess sowie das zu Grunde liegende Datenmodell soweit vereinfacht, das die Funktionalität der Chargenrückverfolgung weiterhin auf die Realität im Unternehmen abgestimmt war. Allerdings konnten eine Vielzahl an Sonderfällen, die grade bei der Verarbeitung von Schweinen auftreten, nicht beachtet werden. Die Vertragssituation zwischen Landwirten und den verarbeitenden Betrieben basiert oft auf mündlichen Absprachen bzw. sind hierfür großzügige Toleranzen in den Verträgen erfasst um nachträgliche Anpassungen beispielsweise bei den Preisen für bestimmte Tiere möglich zu machen. Da der Fokus dieser Arbeit auf der generellen Machbarkeit einer Chargenrückverfolgung mittels der Blockchain Technologie lag, wurden diese Freiheitsgrade nicht weiter betrachtet bzw. in der prototypischen Implementierung beachtet. Solch eine komplexe Wertschöpfungskette wie sie in der Fleischwarenindustrie vorliegt könnte nur schwer die im wissenschaftlichen Kontext dieser Arbeit intendierte notwendige Übertragbarkeit und Reproduzierbarkeit gewährleisten.

\subsection{Ausblick}

Aus wissenschaftlicher Perspektive bietet sich die naheliegendste Fortsetzung dieser Arbeit sicherlich in der Implementierung des vorgestellten Systementwurfs in einem konkreten betrieblichen Umfeld an. Der in dieser Arbeit entwickelte Prototyp kann hierbei als Grundlage zur Erforschung weiterer Prozesse die über eine Blockchain abgebildet werden herangezogen werden. Dabei könnte der gezeigte Systementwurf mit entsprechendem Aufwand für weitere Tierarten, Veterinärinformationen oder Futtermitteldaten erweitert werden. Ebenfalls wäre es denkbar eine vorhandene \acf{iot} Lösung zu integrieren, um Sensordaten aus den Betriebsstätten bzw. während des Transports direkt in die Blockchain einfließen zu lassen. Hierdurch könnte der Informationsgehalt für Aussagen zu einer Charge oder dem gesamten Lebenszyklus eines einzelnen Tieres vom Landwirt bis zum Endkunden noch einmal deutlich erhöht werden. Außerdem bietet sich eine Integration der Blockchaindatenbasis mit vorhandenen \ac{erp}-Systemen an. \ac{erp}-Systeme halten eine große Menge an Stamm- und Bewegungsdaten aus dem betrieblichen Kontext vor. Lassen sich diese Daten mit den Transaktionsdaten des Blockchain Netzwerks verknüpfen eröffnen sich weitere Anwendungsgebiete beispielsweise im Bereich von Business Intelligence.


\newpage
