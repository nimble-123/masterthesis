\section{Problemstellung}

Diese Arbeit beschäftigt sich mit dem Problem der zeitnahen und transparenten Nachverfolgung von Chargen und Einzelprodukten über den gesamten Verlauf der Wertschöpfungskette.\\

Die Fragestellung der Arbeit lautet: Kann die Blockchain Technologie den Prozess der Nachverfolgung von Chargen und/oder individuellen Produkten von der Rohstoffgewinnung bis hin zum letztendlichen Verkauf an den Endverbraucher für alle Teilnehmer der Wertschöpfungskette transparenter und sicherer gestalten?\\

Abbildung \ref{fig:statista-huerden-blockchain-2016} zeigt die Ergebnisse einer Umfrage des Fachmagazins Cofinpro zum Thema \glqq Wo sehen Sie Hürden für die Blockchain-Technologie?\grqq~. So scheinen die mit Abstand größten Einstiegsbarrieren fehlende Standards und rechtliche Regelungen zu sein.\\

\begin{figure}[h!]
	\centering
	\includegraphics[width=0.7\linewidth]{pictures/Statista-Huerden-Blockchain-2016}
	\caption[Statista Blockchain Umfrage]{Cofinpro - Wo sehen Sie Hürden für die Blockchain Technologie? \cite{Cofinpro}}
	\label{fig:statista-huerden-blockchain-2016}
\end{figure}

%Soll die Blockchain Technologie zum Einsatz kommen gibt es offene Fragen. Eine Blockchain ist keine Silberkugel für sämtliche betriebswirtschaftliche Prozesse. Viel mehr kann eine Blockchain als Skalpell dienen um präzise ein bestimmtes Problem zu lösen.\\

%\begin{itemize}
%	\item Technologie ist so neu und frisch verfügbar im Industriekontext
%	\item Ermittlung und Definition möglicher Geschäftsprozesse der Energiewirtschaft
%	\item Vorhandene Lösungen am Markt vergleichen für den Einsatz
%	\item Mehrwert eines DLT-basierten Geschäftsprozesses herausarbeiten
%	\item Spieltheorie neue Geschäftsfelder Blockchain
%\end{itemize}

%Video [Wir und die Blockchain (The Blockchain and Us) (2017) - Deutsche Synchronfassung/German version - YouTube](\url{https://www.youtube.com/watch?v=x2mKDWsNijo})

% konkrete Beschreibung des Problems

\newpage
