\section{Vorläufige Gliederung}
\begin{small}
	1. Einleitung\\
	\noindent\hspace*{10mm}%
	1.1. Motivation\\
	\noindent\hspace*{10mm}%
	1.2. Problemstellung\\
	\noindent\hspace*{10mm}%
	1.3. Lösungsansatz\\
	\noindent\hspace*{10mm}%
	1.4. Struktur der Arbeit\\
	2. Grundlagen Chargenrückverfolgung in der Nahrungsmittelindustrie\\
	\noindent\hspace*{10mm}%
	3.1. ABC\\
	\noindent\hspace*{10mm}%
	3.2. XYZ\\
	\noindent\hspace*{10mm}%
	3.3. ÄÖÜ\\
	3. Grundlagen Blockchain\\
	\noindent\hspace*{10mm}%
	3.1. Definition\\
	%2.1.1. Blockchain\\
	%2.1.2. Tangle\\
	%2.1.3. Hash Graph\\
	%2.1.4. Distributed Ledger\\
	\noindent\hspace*{10mm}%
	3.2. Arten von DLT\\
	%2.2.1. Public\\
	%2.2.2. Private\\
	%2.2.3. Consortium\\
	\noindent\hspace*{10mm}%
	3.3. Abgrenzung zu Cryptocurrencies\\
	\noindent\hspace*{10mm}%
	3.4. Technologischer Aufbau\\
	%2.4.1. Sicherheit\\
	%2.4.1.1. Public-Key Authorization\\
	%2.4.1.2. Hashing Algorithm\\
	%2.4.2. Consensus Algorithm\\ https://www.btc-echo.de/konsens-vs-regierung-warum-bitcoin-keine-demokratie-ist/
	%2.4.3. Dezentralisierung\\
	%2.4.3.1. Peer-to-Peer Netzwerke\\
	%2.4.3.2. Shared Computing\\
	\noindent\hspace*{10mm}%
	3.5. Ausprägungen von DLTs\\
	%2.5.1. Bitcoin Blockchain\\
	%2.5.2. Ethereum Blockchain\\
	%2.5.2.1. Ethereum\\
	%2.5.2.2. Ethereum Enterprise Alliance\\
	%2.5.3. Iota Tangle\\
	%2.5.4. Ripple\\
	%2.5.5. IBM Bluemix\\
	%2.5.6. Microsoft Azure\\
	%2.5.7. SAP Leonardo\\
	%2.5.8. Hyperledger\\
	4. Blockchain Technologie in der Nahrungsmittelindustrie\\
	\noindent\hspace*{10mm}%
	4.1. Funktionale Anforderunggen\\
	%4.1.1. SWOT-Analyse\\
	%4.1.2. Entscheidungsbaum\\
	\noindent\hspace*{10mm}%
	4.2. Nicht-Funktionale Anforderunggen\\
	%4.2.1. Transaktional\\
	%4.2.2. Geschwindigkeit\\
	%4.2.3. Transparenz\\
	%4.2.4. Vertrauen\\
	%4.2.5. Unveränderlichkeit\\
	%4.2.6. Geschäftsregeln\\
	\noindent\hspace*{10mm}%
	4.3. Mehrwerte durch DLT\\
	%4.3.1. Transaktionskosten senken\\
	%4.3.2. Transaktionsgeschwindigkeit erhöhen\\
	%4.3.3. Datenverfügbarkeit über die gesamte Supply Chain erhöhen\\
	%4.3.4. Neue innovative Geschäftsmodelle ermöglichen\\
	5. Prototypische Umsetzung\\
	\noindent\hspace*{10mm}%
	5.1 Environment\\
	%5.2.1. Business Network\\
	%5.2.2. Security\\
	%5.2.3. Cloud Resources\\
	\noindent\hspace*{10mm}%
	5.2 Development\\
	%5.3.1. Tools\\
	%5.3.2. Listings\\
	\noindent\hspace*{10mm}%
	5.3 Deployment\\
	6. Fazit\\
\end{small}

\newpage
