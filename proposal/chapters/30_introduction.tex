\section{Motivation}
Die Welt ist auf dem Weg in das Web 3.0, die Unternehmen wechseln zur Industrie 4.0 und die Cloud kommt demnächst auch in Version 2.0.
%cite
Was am Ende bleibt sind Technologien die sich gegenüber anderen Technologien bewährt haben zum Einen durch ihren Mehrwert in der Forschung und zum Anderen ihren praktischem Nutzen im breiten Feld.
Eine Technologie die bereits heute eine vielzahl von neuen innovativen Ideen im privaten Sektor hervorgebracht hat  - und auch noch weiter werden wird - erobert nach und nach den industriellen Sektor. So erwägen einige der größten Finanzinstitutionen den Einsatz einer sog. Blockchain.
%cite
Die Kernidee hinter einer jeden Blockchain ist es einen Intermediär zu substituieren, der nur eingesetzt wurde um eine neutrale Vertrauensbildung zu ermöglichen. Hier setzt die Distributed Ledger Technology (DLT) mit Mathematik und Kryptographie an. Im Kontrast zum konventionellen Intermediär repräsentiert durch eine dritte Person oder Institution wie z.B. eine Bank oder ein Notar.
Im praktischen Einsatz zeigen sich signifikante Unterschiede von Blockchain zu Blockchain. Es gibt bereits die 3. Generation von DLTs im Consumer Bereich. Durch eine große Menge an Kombinationen in der Parametrisierung und der Interoperabilität einer Blockchain mit vorhandenen digital abgebildeten Geschäftsprozessen lässt sich für den Business Bereich kein schlüsselfertiges Blockchain-Konzept finden. Jeder Use-case der schon heute eine datenbankbasierte Datenhaltung nutzt kann vom Prinzip her auch mit einer Blockchain funktionieren. Es ist allerdings nicht immer gegeben, dass der Einsatz sich lohnt und der Prozess für sich betrachtet an Effizienz gewinnt.

\newpage