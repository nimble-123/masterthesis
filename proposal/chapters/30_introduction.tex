\section{Motivation}
Die Welt ist auf dem Weg in das Web 3.0, die Unternehmen wechseln zur Industrie 4.0 und die Cloud kommt demnächst in Version 2.0.
%cite
Was am Ende bleibt sind Technologien die sich gegenüber anderen Technologien bewährt haben zum Einen durch ihren Mehrwert in der Forschung und zum Anderen ihren praktischem Nutzen im breiten Feld.\\
Eine Technologie die bereits heute eine viel zahl von neuen innovativen Ideen im privaten Sektor hervorgebracht hat  - und auch noch weiter werden wird - erobert nach und nach den industriellen Sektor - Blockchain. So erwägen einige der größten Finanzinstitutionen den Einsatz eines Distributed Ledgers.\\
%cite

“Es ist davon auszugehen, dass wir in ein bis zwei Jahrzehnten wirtschaftlich über Mechanismen miteinander interagieren werden, für die wir bislang weder Konzepte noch Begriffe haben.”\cite[S.~92]{Platzer2014}\\

Die Kernidee hinter jeder Blockhain ist es einen Intermediär zu substituieren, der nur eingesetzt wurde um eine neutrale Vertrauensbildung zu ermöglichen. Hier setzt die Distributed Ledger Technology (DLT) mit Mathematik und Kryptographie an. Im Kontrast zum konventionellen Intermediär repräsentiert durch eine dritte Person oder Institution wie z.B. eine Bank oder einen Notar.\\
Im praktischen Einsatz zeigen sich signifikante Unterschiede von Blockchain zu Blockchain. Es gibt bereits die 3. Generation von DLTs im Consumer Bereich. Durch eine große Menge an Kombinationen in der Parametrisierung und der Interoperabilität einer Blockchain mit vorhandenen digital abgebildeten Geschäftsprozessen lässt sich für den Business Bereich kein schlüsselfertiges Blockchain-Konzept finden.\\

Jeder Use-case der schon heute eine datenbankbasierte Datenhaltung nutzt kann vom Prinzip her auch mit einer Blockchain funktionieren. Es ist allerdings nicht immer gegeben, dass der Einsatz sich lohnt und der Prozess für sich betrachtet an Effizienz gewinnt.\\

Grade die Energiebranche befindet sich seit Jahren im Umbruch. Alte Infrastrukturen, neue Technologien und Wettbewerbsanforderungen lassen sich zum aktuellen Zeitpunkt nicht wirklich effizient in einem System zusammenfassen. Der Energiemarkt wird immer fragmentierter durch kleine Stromerzeuger und neue Möglichkeiten der Stromerzeugung. In diesem Bereich existiert viel Optimierungspotential, das durch neue Technologien genutzt werden kann. Beispielsweise im Stromhandel zwischen Betreibern und Verbrauchern. Einer Umfrage der dena und der privaten Hochschule ESMT Berlin ist zu entnehmen, dass über die Hälfte der Teilnehmer eine weitere Verbreitung von Blockchain Technologie in der Energiewirtschaft für wahrscheinlich halten.\cite[Vgl.]{EnergieAgentur2016}