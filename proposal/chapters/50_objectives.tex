\section{Zielsetzung}

Ziel der Arbeit ist es, die theoretischen Grundlagen der Blockchain Technologie darzulegen und in Kontext mit der Nahrungsmittelindustrie zu bringen, um den Aufbau eines Identitätsmanagements zu evaluieren. Dafür sollen die spezifischen Anforderungen der Branche durch Experteninterviews ermittelt werden und eine Schnittstellenbeschreibung entstehen die es ermöglicht neuen Teilnehmern der Lieferkette unkompliziert am Netzwerk teilzunehmen. Auf dieser Basis soll dann in einer prototypischen Umsetzung die Machbarkeit überprüft werden.\\

Im Vordergrund des Prototyps stehen Aspekte wie Prozesssicherheit, Schutz vor Manipulation durch Teilnehmer und Möglichkeiten der Geheimhaltung von Geschäftsgeheimnissen bei maximaler Transparenz für alle Teilnehmer.\\

Der technische Hintergrund einer Blockchain ist nicht neu. Die einzelnen Komponenten sind bereits heute vielfach erprobt im produktiven Einsatz. \cite{Diffie1976}\cite{Steinmetz2005} Die Kombination zu einer Blockchain ist allerdings neu und aktuell nur in Pilotprojekten für vereinzelte Use-cases zu finden.\\



\newpage
