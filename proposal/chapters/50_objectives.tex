\section{Zielsetzung}

Ziel ist es, die theoretischen Grundlagen der Blockchain Technologie darzulegen und nachzuweisen, ob sie auf die Nahrungsmittelindustrie übertragbar sind, um den Aufbau eines Identitätsmanagements zu evaluieren. Dafür sollen die spezifischen Anforderungen der Branche durch Experteninterviews ermittelt werden und eine erste Schnittstellenbeschreibung entstehen die es ermöglicht neuen Teilnehmern der Lieferkette unkompliziert am Netzwerk teilzunehmen. Auf dieser Basis soll dann in einer prototypischen Umsetzung die Machbarkeit der Anwendung von Blockchain Technologie in der Nahrungsmittelindustrie überprüft bzw. evaluiert werden.\\

Im Vordergrund des Prototyps stehen Aspekte wie Prozesssicherheit, Schutz vor Manipulation durch Teilnehmer und Externe wie auch Möglichkeiten der Geheimhaltung von Geschäftsgeheimnissen bei maximaler Transparenz für alle Teilnehmer.\\

% Der technische Hintergrund einer Blockchain ist nicht neu. Die einzelnen Komponenten sind bereits heute vielfach erprobt im produktiven Einsatz. \cite{Diffie1976}\cite{Steinmetz2005} Die Kombination zu einer Blockchain ist allerdings neu und aktuell nur in Pilotprojekten für vereinzelte Use-cases zu finden.\\

\newpage
