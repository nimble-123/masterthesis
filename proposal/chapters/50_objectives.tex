\section{Zielsetzung}

Jeder Use-case der schon heute eine datenbankbasierte Datenhaltung nutzt kann vom Prinzip her auch mit einer Blockchain funktionieren. Es ist allerdings nicht immer gegeben, dass der Einsatz sich lohnt und der Prozess für sich betrachtet an Effizienz gewinnt.\\

Grade die Energiebranche befindet sich seit Jahren im Umbruch. Alte Infrastrukturen, neue Technologien und Wettbewerbsanforderungen lassen sich zum aktuellen Zeitpunkt nicht wirklich effizient in einem System zusammenfassen. Der Energiemarkt wird immer fragmentierter durch kleine Stromerzeuger und neue Möglichkeiten der Stromerzeugung. In diesem Bereich existiert viel Optimierungspotential, das durch neue Technologien genutzt werden kann. Beispielsweise im Stromhandel zwischen Betreibern und Verbrauchern.

Daher sollen in dieser Masterarbeit folgende Punkte behandelt werden.

\begin{itemize}
	\item Kriterien ausarbeiten um Prozesse zu finden die für DLT geeignet sind
	\item Prototypische Umsetzung eines DLT-basierten Geschäftsprozess
	\item Bewertung der potentiellen Implikationen beim Einsatz eines DLT
\end{itemize}

\newpage