\section{Ziele}

Ziel dieser Arbeit ist es, durch Entwicklung und Evaluation eines Prototyps die Möglichkeiten und Grenzen der Blockchain Technologie im Kontext der Chargenrückverfolgung in der Fleischwarenindustrie zu überprüfen. Dabei sollen die dafür nötigen Daten und Informationen ermittelt und in den System Entwurf eingearbeitet werden. Außerdem ist angestrebt aus der vielzahl von unterschiedlichen Implementierungen einer Blockchain die Ausprägung zu identifizieren, welche für die spezifischen Anforderungen der Fleischwarenindustrie ideal erscheint.

Der enstandene Prototyp soll beim Praxispartner Westfleisch SCE mbH als Entscheidungshilfe für eine zukünftige Innovationsstrategie zur Optimierung der Lieferkette dienen.

% Ziel dieser Arbeit ist es, die theoretischen Grundlagen der \textit{Blockchain Technologie} darzulegen und nachzuweisen, ob sie auf die Fleischwarenindustrie übertragbar sind, um den Aufbau eines Systems zur Chargenrückverfolgung zu evaluieren. Dafür sollen die spezifischen Anforderungen der Branche durch Experteninterviews ermittelt werden und eine erste Schnittstellenbeschreibung entstehen die es ermöglicht neuen Teilnehmern der Lieferkette unkompliziert am Netzwerk teilzunehmen. Auf dieser Basis soll dann in einer prototypischen Umsetzung die Machbarkeit der Anwendung von \textit{Blockchain Technologie} in der Nahrungsmittelindustrie überprüft bzw. evaluiert werden.\\

% Im Vordergrund des Prototyps stehen Aspekte wie Prozesssicherheit, Schutz vor Manipulation durch Teilnehmer und Externe wie auch Möglichkeiten der Geheimhaltung von Geschäftsgeheimnissen bei maximaler Transparenz für alle Teilnehmer.\\

% Der technische Hintergrund einer Blockchain ist nicht neu. Die einzelnen Komponenten sind bereits heute vielfach erprobt im produktiven Einsatz. \citep{Diffie1976}\citep{Steinmetz2005} Die Kombination zu einer Blockchain ist allerdings neu und aktuell nur in Pilotprojekten für vereinzelte Use-cases zu finden.\\

% \newpage
